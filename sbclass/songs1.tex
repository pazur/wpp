\song[lyricsyear={1797},alt={Pieśń Legionów Polskich we Włoszech},lyrics={Józef Wybicki}]
{Mazurek Dąbrowskiego}
\begin{info}Pierwotnie hymn, jako Pieśń Legionów Polskich we Włoszech, został napisany przez Józefa Wybickiego. Autor melodii opartej na motywach ludowego mazurka (właściwie mazura) jest nieznany. Pieśń powstała w dniach 16-19 lipca 1797 we włoskim miasteczku Reggio nell'Emilia w Republice Cisalpińskiej (w dzisiejszych Włoszech). Pierwszy raz została wykonana publicznie 20 lipca 1797 roku. Tekst ogłoszono po raz pierwszy w Mantui w lutym 1799 w gazetce "Dekada Legionowa".\end{info}

\begin{lyrics}[longestline={Przejdziem Wisłę, przejdziem Wartę,}]

\firstwords{Jeszcze Polska nie zginęła},\\*
Kiedy my żyjemy.\\*
Co nam obca przemoc wzięła,\\*
Szablą odbierzemy.

\begin{chorus}
\chorusfirstwords{Marsz, marsz, Dąbrowski},\\*
Z ziemi włoskiej do Polski.\\*
Za twoim przewodem\\*
Złączym się z narodem.
\end{chorus}

Przejdziem Wisłę, przejdziem Wartę,\\*
Będziem Polakami.\\*
Dał nam przykład Bonaparte,\\*
Jak zwyciężać mamy.

\chorusref

Jak Czarniecki do Poznania\\*
Po szwedzkim zaborze,\\*
Dla ojczyzny ratowania\\*
Wrócim się przez morze.

\chorusref

Już tam ojciec do swej Basi\\*
Mówi zapłakany —\\*
Słuchaj jeno, pono nasi\\*
Biją w tarabany.

\chorusref
\end{lyrics}



\song
{Bogurodzica}
\begin{info}Jest to najstarsza polska pieśń religijna, a zarazem najstarsza pieśń bojowa polskiego rycerstwa. Pochodzi prawdopodobnie z końca XIII w. Śpiewana była przez wojska polskie przed bitwami z Krzyżakami: pod Grunwaldem (1410 r.) i pod Dąbkami koło Nakła nad Notecią (1431 r.), przed bitwą z księciem litewskim Świdrygiełłą pod Wiłkomierzem (1435 r.), a także podczas koronacji królów i przy wydawaniu ważniejszych dekretów państwowych. Można ją więc określić jako ówczesny hymn narodowy. Funkcję tę utraciła jednakże w drugiej połowie XVI w.\end{info}

\begin{lyrics}[longestline={U twego Syna Gospodzina, Matko zwolena, Maryja!}]

\firstwords{Bogurodzica Dziewica, Bogiem sławiena Maryja},\\*
U twego Syna Gospodzina, Matko zwolena, Maryja!\\*
Zyszczy nam, spuści nam.\\*
Kyrie eleison.

Twego dziela Krzciciela, Bożyszcze,\\*
Usłysz głosy, napełń myśli człowiecze.\\*
Słysz modlitwę, jąż nosimy,\\*
A dać raczy, Jegoż prosimy:\\*
A na świecie zbożny pobyt,\\*
Po żywocie rajski przebyt!\\*
Kyrie eleison.
\end{lyrics}



\song[lyricsyear={1944},musicyear={1944},music={Alfred L. Schütz},lyrics={Feliks Konarski}]
{Czerwone maki}

\begin{lyrics}[longestline={Czerwieńsze będą, bo z polskiej wzrosły krwi.}]

\firstwords{Czy widzisz te gruzy na szczycie}?\\*
Tam wróg twój się kryje jak szczur!\\*
Musicie, musicie, musicie\\*
Za kark wziąć i strącić go z chmur!\\*
I poszli szaleni zażarci,\\*
I poszli zabijać i mścić,\\*
I poszli jak zawsze uparci,\\*
Jak zawsze za honor się bić.

\begin{chorus}
\chorusfirstwords{Czerwone maki na Monte Cassino}\\*
Zamiast rosy piły polską krew.\\*
Po tych makach szedł żołnierz i ginął,\\*
Lecz od śmierci silniejszy był gniew.\\*
Przejdą lata i wieki przeminą.\\*
Pozostaną ślady starych dni\\*
I tylko maki na Monte Cassino\\*
Czerwieńsze będą, bo z polskiej wzrosły krwi.
\end{chorus}

Runęli przez ogień ,straceńcy,\\*
niejeden z nich dostał i padł,\\*
jak ci z Somosierry szaleńcy,\\*
Jak ci spod Rokliny przed lat.\\*
Runęli impetem szalonym,\\*
I doszli . I udał się szturm.\\*
I sztandar swój biało czerwony\\*
Zatknęli na gruzach wśród chmur,

\chorusref

Czy widzisz ten rząd białych krzyży?\\*
Tam Polak z honorem brał ślub.\\*
Idź naprzód, im dalej, im wyżej,\\*
Tym więcej ich znajdziesz u stóp.\\*
Ta ziemia do Polski należy,\\*
Choć Polska daleko jest stąd,\\*
Bo wolność krzyżami się mierzy,\\*
Historia ten jeden ma błąd.

\chorusref
\end{lyrics}



\song[lyricsyear={1939},lyrics={Julia Ryczer}]
{Dnia pierwszego września}
\begin{info}Chyba najpopularniejsza spośród piosenek stanowiących w latach okupacji hitlerowskiej repertuar grajków i śpiewaków, którzy produkowali się na ulicach, podwórkach, w tramwajach i w pociągach. \end{info}

\begin{lyrics}[longestline={Jeszcze Pan Bóg pomści taką straszną zbrodnie.}]

\firstwords{Dnia pierwszego września, roku pamiętnego}\\*
Wróg napadł na Polskę z kraju sąsiedniego

Najwięcej się uwziął na naszą Warszawę\\*
Warszawo kochana tyś jest miasto krwawe

Kiedyś byłaś piękna bogata wspaniała\\*
Teraz tylko kupa gruzów pozostała

Domy popalone, szpitale zburzone\\*
Gdzie się mają podziać ludzie poranione

Lecą bomby z nieba brak jest ludziom chleba\\*
Nie tylko od bomby umrzeć będzie trzeba

Gdy biedna Warszawa w gruzach pozostała\\*
To biedna Warszawa poddać się musiała

I tak się broniła całe trzy tygodnie\\*
Jeszcze Pan Bóg pomści taką straszną zbrodnie.
\end{lyrics}



\song[lyricsyear={1942},alt={Kołysanka leśna},lyrics={\footnote{Autorstwo przypisywane bywa Stanisławowi Magierskiemu z Lublina, Bronisławowi Królowi z lwowskiej grupy poetyckiej ,,Żagiew'' oraz Krystynie Krahelskiej.}}]
{Dziś do ciebie przyjść nie mogę}
\begin{info}Piosenka partyzancka z przełomu 1942–1943. W okupowany kraj popłynęła z Lubelszczyzny. Do dzisiaj nie ustalono twórcy tych niezwykle pięknych strof i oryginalnej melodii. Domniemane autorstwo przypisywane bywa Stanisławowi Magierskiemu (1904–1957) z Lublina, Bronisławowi Królowi (1916) z lwowskiej grupy poetyckiej „Żagiew” i Krystynie Krahelskiej (1914–1944).\end{info}

\begin{lyrics}[longestline={Dziś do ciebie przyjść nie mogę,}]

\firstwords{Dziś do ciebie przyjść nie mogę},\\*
Zaraz idę w nocy mrok.\\*
Nie wyglądaj za mną oknem,\\*
W mgle utonie próżno wzrok.

Po cóż ci, kochanie, wiedzieć,\\*
Że do lasu idę spać,\\*
Dłużej tu nie mogę siedzieć,\\*
Na mnie czeka leśna brać.

Księżyc zaszedł, hen, za lasem,\\*
We wsi gdzieś szczekają psy,\\*
A nie pomyśl sobie czasem,\\*
Że do innej tęskno mi.

Kiedy wrócę, dziś do ciebie,\\*
Może w dzień, a może w noc,\\*
Dobrze będzie nam jak w niebie,\\*
Pocałunków dasz mi moc.

Gdy nie wrócę, niechaj z wiosną\\*
Rolę moją sieje brat,\\*
Kości moje mchem porosną\\*
I użyźnią ziemi szmat.

W pole wyjdź pewnego ranka,\\*
Na snop żyta dłonie złóż\\*
I ucałuj, jak kochanka.\\*
Ja żyć będę w kłosach zbóż.
\end{lyrics}



\song[lyricsyear={prawd. 1918},alt={Maki},music={Stanisław Niewiadomski},lyrics={Kornel Makuszyński}]
{Ej dziewczyno, ej niebogo}

\begin{lyrics}[longestline={niech mnie zabiera, zabiera, zabiera.}]

\firstwords{Ej dziewczyno, ej niebogo},\\*
jakieś wojsko pędzi drogą,\\*
\markverse[marktext={bis}]{skryj się za ścianę}\\*
\markverse[marktext={bis}]{skryj się, skryj.}\\*
Ja myślałem, że to maki,\\*
że ogniste lecą ptaki,\\*
\markverse[marktext={bis}]{a to ułany, ułany, ułany.}

Serce weźmie, w dal pobiegnie,\\*
potem w krwawym boju legnie,\\*
\markverse[marktext={bis}]{zostaniesz wdową!}\\*
\markverse[marktext={bis}]{Strzeż się, strzeż!}\\*
Łez ja po nim nie uronię,\\*
własną piersią go zasłonię,\\*
\markverse[marktext={bis}]{Bóg go zachowa, zachowa, zachowa.}

Strzeż się tego, co na przedzie,\\*
co na karym koniu jedzie,\\*
\markverse[marktext={bis}]{oficyjera, oficyjera,}\\*
\markverse[marktext={bis}]{strzeż się, strzeż!}\\*
Jeśli mu się wydam miła,\\*
to nie będę się broniła,\\*
\markverse[marktext={bis}]{niech mnie zabiera, zabiera, zabiera.}
\end{lyrics}



\song[lyricsyear={ok. 1840},music={Wolfgang A. Mozart\footnote{fragment opery Don Juan}},lyrics={Gustaw Ehrenberg}]
{Gdy naród do boju}
\begin{info}Bojowa pieśń chłopska i hymn ruchu ludowego oraz rewolucyjnego, żywa zwłaszcza w szeregach lewicy. Pierwodruk tekstu ukazał się pt. Szlachta w roku 1831 w tomie Dźwięki z minionych lat z 1848 r.\end{info}

\begin{lyrics}[longestline={Gdy naród zawołał: ,,Umrzem lub zwyciężym!''}]

\firstwords{Gdy naród do boju wystąpił z orężem},\\*
Panowie o czynszach radzili,\\*
Gdy naród zawołał: ,,Umrzem lub zwyciężym!''\\*
Panowie w stolicy bawili.

\begin{chorus}
\chorusfirstwords{O, cześć wam, panowie magnaci},\\*
Za naszą niewolę, kajdany,\\*
O, cześć wam, książęta, hrabiowie, prałaci,\\*
Za kraj nasz krwią bratnią zbryzgany.
\end{chorus}

Armaty pod Stoczkiem zdobywała wiara\\*
Rękami czarnymi od pługa.\\*
Panowie w stolicy kurzyli cygara,\\*
Radzili o braciach zza Buga.

\chorusref

Wszak waszym był synem ów niecny kunktator\\*
Co wzbudzał przed wrogiem obawę,\\*
l wódz ten naczelny, pobożny dyktator,\\*
l zdrajca, co sprzedał Warszawę.

\chorusref

Lecz kiedy wybije godzina powstania,\\*
Magnatom lud ucztę zgotuje,\\*
Muzykę piekielną zaprosi do grania,\\*
A szlachta niech wtedy tańcuje.

\chorusref
\end{lyrics}



\song[lyricsyear={1914},lyrics={Feliks Gwiżdż}]
{Przybyli ułani pod okienko}

\begin{lyrics}[longestline={,,Otwieraj! Nie bój się to czwartacy!'' bis}]

\markverse[marktext={bis}]{\firstwords{Przybyli ułani pod okienko},}\\*
\markverse[marktext={bis}]{Pukają, wołają: ,,puść panienko!''}

\markverse[marktext={bis}]{Zaświecił miesiączek do okienka,}\\*
\markverse[marktext={bis}]{W koszulce stanęła w nim panienka.}

\markverse[marktext={bis}]{,,O Jezu, a cóż to za wojacy?''}\\*
,,Otwieraj! Nie bój się to czwartacy!'' bis

\markverse[marktext={bis}]{,,O Jezu! A dokąd Bóg prowadzi?''}\\*
\markverse[marktext={bis}]{,,Warszawę odwiedzić byśmy radzi.''}

\markverse[marktext={bis}]{Gdy zwiedzim Warszawę, już nam pilno}\\*
\markverse[marktext={bis}]{Zobaczyć to stare nasze Wilno.}

\markverse[marktext={bis}]{A z Wilna już droga jest gotowa,}\\*
\markverse[marktext={bis}]{Prowadzi prościutko aż do Lwowa.}

\markverse[marktext={bis}]{''O Jezu, a cóż to za mizeria?''}\\*
\markverse[marktext={bis}]{''Otwórz no, panienko! Kawaleria.''}

\markverse[marktext={bis}]{Przyszliśmy napoić nasze konie,}\\*
\markverse[marktext={bis}]{Za nami piechoty pełne błonie.''}

\markverse[marktext={bis}]{''O Jezu! A cóż to za hołota?''}\\*
\markverse[marktext={bis}]{''Otwórz panienko! To piechota!''}

\markverse[marktext={bis}]{Panienka otwierać podskoczyła,}\\*
\markverse[marktext={bis}]{Żołnierzy do środka zaprosiła.}
\end{lyrics}



\song[music={Jerzy Wasowski},lyrics={Bronisław Brok}]
{Po ten kwiat czerwony}

\begin{lyrics}[longestline={Po ten kwiat, po ten kwiat czerwony,}]

\firstwords{Żołnierz dziewczynie nie skłamie},\\*
Chociaż nie wszystko jej powie,\\*
Żołnierz zarzuci broń na ramię,\\*
Wróci, - to resztę dopowie.

\begin{chorus}
\chorusfirstwords{Wstęgą szos, miedzą pól złoconych},\\*
krętą ścieżką poprzez las,\\*
Po ten kwiat, po ten kwiat czerwony,\\*
- Skoro przyszedł na to czas.\\*
Po ten kwiat, po ten kwiat czerwony,\\*
Skoro przyszedł na to czas.
\end{chorus}

Idę, wrócę, o nic nie pytaj dziś\\*
Przyjdę , nie zasmucę,\\*
powiem po com musiał iść

\chorusref

Wstęgą szos, miedzą pól złoconych...

Dla tych co wiernie czekają\\*
Będą żołnierze śpiewali\\*
O tym, jak pięknie zakwitają\\*
Kwiaty czerwieńsze od malin.

\chorusref

Wstęgą szos, miedzą pól złoconych...
\end{lyrics}



\song
{Hej, sokoły}

\begin{lyrics}[longestline={Hej, tam gdzieś znad czarnej wody,}]

\firstwords{Hej, tam gdzieś znad czarnej wody},\\*
Wsiada na koń kozak młody,\\*
Czule żegna się z dziewczyną,\\*
Jeszcze czulej z Ukrainą.

\begin{chorus}
\chorusfirstwords{Hej, hej, hej sokoły},\\*
Omijajcie góry, lasy, doły,\\*
Dzwoń, dzwoń, dzwoń dzwoneczku,\\*
Mój stepowy skowroneczku.
\end{chorus}

Żal, żal za dziewczyną,\\*
Za zieloną Ukrainą,\\*
Żal, żal serce płacze,\\*
Żal, że już jej nie zobaczę.

\chorusref

Ona biedna tam została,\\*
Przepióreczka moja mała,\\*
A ja tutaj, w obcej strome,\\*
Dniem i nocą tęsknię do niej.

\chorusref
\end{lyrics}



\song[lyricsyear={wrzesień 1942},music={\footnote{na melodię Co użyjem, to dla nas}},lyrics={Anna Jachnina}]
{Siekiera, motyka}
\begin{info}Piosenka popularna na ulicach Warszawy w czasie okupacji niemieckiej podczas II wojny światowej.\end{info}

\begin{lyrics}[longestline={Siekiera, motyka, bimber, szklanka,}]

\firstwords{Siekiera, motyka, bimber, szklanka},\\*
w nocy nalot, w dzień łapanka,\\*
siekiera, motyka, światło, prąd,\\*
kiedyż oni pójdą stąd.

Siekiera, motyka, tramwaj, buda,\\*
Każdy zwiewa gdzie się uda,\\*
Siekiera, motyka, igła, nić,\\*
Już nie mamy gdzie się skryć.

Już nie mamy gdzie się skryć,\\*
Szwaby nam nie dają żyć.\\*
Ich kultura nie zabrania\\*
Robić takie polowania

Siekiera, motyka, piłka, linka,\\*
tu Oświęcim, tam Treblinka,\\*
siekiera, motyka, światło, prąd,\\*
drałuj, draniu, wreszcie stąd.

Siekiera, motyka, styczeń, luty,\\*
Hitler z Ducem gubią buty,\\*
siekiera, motyka, linka, drut,\\*
już pan malarz jest kaput.

Jak tu być i o czym śnić,\\*
Hycle nam nie dają żyć.\\*
Po ulicach gonią wciąż,\\*
patrzą, kogo jeszcze wziąć.

Siekiera, motyka, piłka, alasz,\\*
przegrał wojnę głupi malarz,\\*
siekiera, motyka, piłka, nóż,\\*
przegrał wojnę już, już, już.
\end{lyrics}



\song[lyricsyear={marzec 1915},music={Zygmunt Pomarański},lyrics={Wacław Biernacki}]
{Pieśn o wodzu miłym}
\begin{info}Kasztanka – klacz podarowana Józefowi Piłsudskiemu w majątku czaple Małe 9 sierpnia 1914 r. Była pierwszym stałym i znanym przez wszystkich żołnierzy wierzchowcem Komendanta.\end{info}

\begin{lyrics}[longestline={Masz wierniejszych niż stal chłodna}]

\firstwords{Jedzie, jedzie na Kasztance},\\*
\markverse[marktext={bis}]{Siwy strzelca strój.}\\*
Hej, hej, Komendancie,\\*
Miły Wodzu mój!

Gdzie szabelka twa ze stali?\\*
\markverse[marktext={bis}]{Przecież idziem w bój.}\\*
Hej, hej, Komendancie,\\*
Miły Wodzu mój!

Gdzie twój mundur jeneralski\\*
\markverse[marktext={bis}]{Złotem wyszywany?}\\*
Hej, hej, Komendancie,\\*
Wodzu kochany!

Masz wierniejszych niż stal chłodna\\*
\markverse[marktext={bis}]{Młodych strzelców rój!}\\*
Hej, hej, Komendancie,\\*
Miły Wodzu mój!

Nad lampasy i czerwienie\\*
\markverse[marktext={bis}]{Wolisz strzelca strój!}\\*
Hej, hej, Komendancie,\\*
Miły Wodzu mój!

Ale pod tą szarą bluzą\\*
\markverse[marktext={bis}]{Serce ze złota!}\\*
Hej, hej, Komendancie,\\*
Serce ze złota!

Ale błyszczą groźną wolą\\*
\markverse[marktext={bis}]{Królewskie oczy!}\\*
Hej, hej, Komendancie,\\*
Królewskie oczy!

Pójdzie z tobą po zwycięstwo\\*
\markverse[marktext={bis}]{Młodych strzelców rój!}\\*
Hej, hej, Komendancie,\\*
Miły Wodzu mój!
\end{lyrics}



\song[lyricsyear={1914?},lyrics={\footnote{Autorstwo czasem przypisywane jest Feliksowi Gwiżdżowi.}}]
{Wojenka}
\begin{info}Niezwykle popularna piosenka legionowa, do tej pory anonimowa. Nie notowana przez śpiewniki żołnierskie z czasów I wojny światowej. Wydrukowana po raz pierwszy w śpiewniku Szula.\end{info}

\begin{lyrics}[longestline={Wojenko, wojenko, co za moc jest w tobie}]

\firstwords{Wojenko, wojenko, cóżeś ty za pani},\\*
\begin{markverses}[marktext={x2}]%
że za tobą idą, że za tobą idą\\*
chłopcy malowani?
\end{markverses}

Chłopcy malowani, sami wybierani,\\*
\begin{markverses}[marktext={x2}]%
wojenko, wojenko, wojenko, wojenko,\\*
cóżeś ty za pani?
\end{markverses}

Na wojence ładnie, kto Boga uprosi,\\*
\begin{markverses}[marktext={x2}]%
żołnierze strzelają, żołnierze strzelają,\\*
Pan Bóg kule nosi.
\end{markverses}

Lecą kule, lecą kule żwawo,\\*
\begin{markverses}[marktext={x2}]%
która cię dogoni, która cię dogoni,\\*
to zapłacisz krwawo.
\end{markverses}

Wojenko, wojenko, cóżeś tak szalona,\\*
\begin{markverses}[marktext={x2}]%
kogo ty pokochasz, kogo ty pokochasz,\\*
jeśli nie leguna.
\end{markverses}

Wojenko, wojenko, co za moc jest w tobie,\\*
\begin{markverses}[marktext={x2}]%
kogo ty pokochasz, kogo ty pokochasz,\\*
w zimnym leży grobie.
\end{markverses}
\end{lyrics}



\song
{O mój rozmarynie}
\begin{info}Jedna z najpopularniejszych polskich pieśni wojskowych z czasów I wojny światowej i wojny polsko - bolszewickiej.W śpiewnikach pojawia się od roku 1915, po raz pierwszy wydrukowana została w zbiorze Żołnierskie piosenki obozowe Adama Zagórskiego. Dwie zwrotki dopisał Wacław Denhoff-Czarnocki, a jedna powstała w 4 Pułku Legionów, także w późniejszych czasach dopisywano nowe zwrotki. Jednak autorzy większości tekstu są anonimowi.\end{info}

\begin{lyrics}[longestline={hej, tam kule świszczą i bagnety błyszczą,}]

\markverse[marktext={bis}]{\firstwords{O, mój rozmarynie, rozwijaj się},}\\*
\begin{markverses}[marktext={bis}]%
pójdę do dziewczyny, pójdę do jedynej,\\*
zapytam się,
\end{markverses}

\markverse[marktext={bis}]{A jak mi odpowie — nie kocham cię,}\\*
\begin{markverses}[marktext={bis}]%
ułani werbują, strzelcy maszerują,\\*
zaciągnę się,
\end{markverses}

\markverse[marktext={bis}]{Dadzą mi buciki z ostrogami,}\\*
\begin{markverses}[marktext={bis}]%
i siwy kabacik, i siwy kabacik,\\*
z wyłogami,
\end{markverses}

\markverse[marktext={bis}]{Dadzą mi konika cisawego,}\\*
\begin{markverses}[marktext={bis}]%
i ostrą szabelkę, i ostrą szabelkę,\\*
do boku mego,
\end{markverses}

\markverse[marktext={bis}]{Dadzą mi uniform popielaty,}\\*
\begin{markverses}[marktext={bis}]%
ażebym nie tęsknił, ażebym nie tęsknił,\\*
do swojej chaty,
\end{markverses}

\markverse[marktext={bis}]{Dadzą mi manierkę z gorzałczyną,}\\*
\begin{markverses}[marktext={bis}]%
ażebym nie tęsknił, ażebym nie tęsknił\\*
za dziewczyną,
\end{markverses}

\markverse[marktext={bis}]{A kiedy już wyjdę na wiarusa,}\\*
\begin{markverses}[marktext={bis}]%
pójdę do dziewczyny, pójdę do jedynej,\\*
po całusa,
\end{markverses}

\markverse[marktext={bis}]{A gdy mi odpowie — nie wydam się,}\\*
\begin{markverses}[marktext={bis}]%
hej, tam kule świszczą i bagnety błyszczą,\\*
poświęcę się,
\end{markverses}

\markverse[marktext={bis}]{Pójdziemy z okopów na bagnety,}\\*
\begin{markverses}[marktext={bis}]%
bagnet mnie ukłuje, śmierć mnie pocałuje,\\*
ale nie ty,
\end{markverses}

\markverse[marktext={bis}]{A gdy mnie przyniosą z raną w boku,}\\*
\begin{markverses}[marktext={bis}]%
wtedy pożałujesz, wtedy pożałujesz,\\*
z łezką w oku,
\end{markverses}

\markverse[marktext={bis}]{Za tę naszą ziemię skąpaną we krwi,}\\*
\begin{markverses}[marktext={bis}]%
za naszą niewolę, za nasze kajdany,\\*
za wylane łzy,
\end{markverses}
\end{lyrics}



\song[lyricsyear={1831},music={Karol Kurpiński},lyrics={Karol Sienkiewicz\footnote{Autorem tekstu oryginalnego w języku francuskim jest Kazimierz Delaińgne.}},beginonleft=true]
{Warszawianka}
\begin{info}Casimir Francois Delavigne (1793–1843) był narodowym poetą francuskim. Pod wrażeniem powstania listopadowego napisał w 1831 r. wiersz „La Varsovienne“ („Warszawianka”). Melodię skomponował, po zapoznaniu się z przekładem Karola Sienkiewicza (1793–1869), wybitny kompozytor i dyrygent operowy – Karol Kurpiński (1785–1857), autor „Krakowiaków i Górali”. Warto wspomnieć, że Delavigne przybrał po wybuchu powstania listopadowego imię Casimir (Kazimierz), aby zamanifestować swoją solidarność z Polakami.\end{info}

\begin{lyrics}[multicol=true, longestline={Niech krwią zlane w bojach srogich,}]

\firstwords{Oto dziś dzień krwi i chwały},\\*
Oby dniem wskrzeszenia był.\\*
W gwiazdę Polski Orzeł Biały\\*
Patrząc lot swój w niebo wzbił;\\*
Słońcem lipca podniecany,\\*
Woła na nas z górnych stron:\\*
Powstań, Polsko, skrusz kajdany,\\*
Dziś twój tryumf albo zgon.

\begin{chorus}
\chorusfirstwords{Hej, kto Polak, na bagnety}!\\*
Żyj swobodo, Polsko żyj.\\*
Takiem hasłem cnej podniety,\\*
Trąbo nasza, wrogom grzmij.
\end{chorus}

Na koń! Woła kozak mściwy,\\*
Karać bunty polskich rot!\\*
Bez Bałkanów są ich niwy\\*
Wszystko jeden zgniecie grot!\\*
Stój, za Bałkan pierś ta stanie,\\*
Car wasz marzy płonny łup,\\*
Z wrogów naszych nie zostanie\\*
Na tej ziemi chyba trup.

\chorusref

Droga Polsko! Dzieci twoje\\*
Dziś szczęśliwszych doszły chwil\\*
Od tych sławnych, gdy ich boje\\*
Wieńczył Kremlin, Tybr i Nil,\\*
Lat dwadzieścia nasze męże\\*
Los po obcych grodach siał,\\*
Dziś, o matko, kto polęże,\\*
Na twem łonie będzie spał.

\chorusref

Wstań Kościuszko! Ugodź w serca.\\*
Co litością mamić śmią,\\*
Znałże litość ów morderca,\\*
Który Pragę zalał krwią.\\*
Niechaj krew tę krwią dziś spłaci,\\*
Niech nią zrosi grunt zły gość,\\*
Laur męczeński naszej braci\\*
Bujniej będzie po niej rość.

\chorusref

Tocz, Polaku, bój zacięty,\\*
Uledz musi dumny car,\\*
Pokaż jemu pierścień święty,\\*
Nieulękłych Polek dar.\\*
Niech to godło ślubów drogich\\*
Wrogom naszym wróży grób,\\*
Niech krwią zlane w bojach srogich,\\*
Nasz z wolnością świadczy ślub.

\chorusref

O, Francuzi! Czyż bez ceny\\*
Rany nasze dla was są!\\*
Z pod Marengo, Wagram, Jeny,\\*
Drezna, Lipska, Waterlo?\\*
Świat was zdradza — my dotrwali,\\*
Śmierć czy tryumf — my gdzie wy.\\*
Bracia! My wam krew dawali,\\*
Dziś wy dla nas nic — prócz łzy.

\chorusref

Wy przynajmiej, coście legli,\\*
W obcych krajach, za kraj swój,\\*
Bracia nasi z grobów zbiegli,\\*
Błogosławcie bratni bój.\\*
Bo zwyciężyć my gotowi\\*
Z trupów naszych tamę wznieść,\\*
By krok spóźnić olbrzymowi,\\*
Co chce światu pęta nieść.

\chorusref

Grzmijcie bębny, ryczcie działa,\\*
Dalej! Dzieci, w gęsty szyk!\\*
Wiedzie hufce wolność, chwała,\\*
Tryumf błyska w ostrzu pik.\\*
Leć, nasz Orle, w górnym pędzie,\\*
Sławie, Polsce, światu służ!\\*
Kto przeżyje, wolnym będzie;\\*
Kto umiera, wolnym już!

\chorusref
\end{lyrics}



\song
{Ułani, ułani}

\begin{lyrics}[longestline={Nie ma takiej wioski, nie ma takiej chatki,}]

\firstwords{Ułani, ułani, malowane dzieci},\\*
niejedna panienka za wami poleci.

\begin{chorus}
\chorusfirstwords{Hej, hej, ułani, malowane dzieci},\\*
niejedna panienka za wami poleci.
\end{chorus}

Niejedna panienka i niejedna wdowa,\\*
zobaczy ułana, kochać by gotowa.

\chorusref

Babcia umierała, jeszcze się pytała:\\*
czy na tamtym świecie, ułani, będziecie?

\chorusref

Nie ma takiej wioski, nie ma takiej chatki,\\*
gdzie by nie kochały ułana mężatki.

\chorusref

Nie ma takiej chatki ani przybudówki,\\*
gdzie by nie kochały ułana Żydówki.

\chorusref

Jedzie ułan, jedzie, konik pod nim pląsa,\\*
czapkę ma na bakier i podkręca wąsa,

\chorusref

Jedzie ułan, jedzie, szablą pobrzękuje,\\*
uciekaj, dziewczyno, bo cię pocałuje.

\chorusref

A niech pocałuje, nikt mu nie zabrania,\\*
wszak on swoją piersią Ojczyznę osłania.

\chorusref
\end{lyrics}



\song
{Idzie żołnierz borem, lasem}
\begin{info}Jest to jedna z najstarszych piosenek żołnierskich. Powstała najprawdopodobniej w czasach napoleońskich, lecz geneza jej powstania może także sięgać wcześniejszych lat. Najwcześniejszy jej zapis pochodzi z 1584 r., ale niektórzy badacze przypuszczają, że jej narodziny sięgają czasów bitwy pod Warną (1444 r.)\end{info}

\begin{lyrics}[longestline={Z głodu czasem, z głodu czasem}]

\firstwords{Idzie żołnierz}\\*
Borem lasem, borem lasem\\*
Przymierając\\*
Z głodu czasem, z głodu czasem

Suknia na nim\\*
Nie blakuje, nie blakuje\\*
Wiatr dziurami\\*
Przelatuje, przelatuje

Chustka czarna\\*
Jest za pasem, jest za pasem\\*
Ale i tej\\*
Pusto czasem, pusto czasem

Chociaż żołnierz\\*
Obszarpany, obszarpany\\*
Przecież ujdzie\\*
Między pany, między pany

Trzeba by go\\*
Obdarować obdarować\\*
Soli, chleba\\*
Nie żałować nie żałować

Wtenczas żołnierza\\*
Szanują, szanują\\*
Kiedy trwogę\\*
Na się czują, na się czują

Zapłaćże mu\\*
Jezu z nieba, Jezu z nieba\\*
Boć go pilna\\*
Jest potrzeba, jest potrzeba
\end{lyrics}



\song[lyricsyear={1974},lyrics={Jacek Kaczmarski}]
{Obława}
\begin{info} Jest to wolne tłumaczenie piosenki Włodzimierza Wysockiego Охота на волков.\end{info}

\begin{lyrics}[longestline={Bo z trzema na raz walczy psami i trzech ran na raz krwawi.}]

\firstwords{Skulony w jakiejś ciemnej jamie smaczniem sobie spał}\\*
I spały małe wilczki dwa - zupełnie ślepe jeszcze\\*
Wtem stary wilk przewodnik, co życie dobrze znał\\*
Łeb podniósł, warknął groźnie, aż mną szarpnęły dreszcze\\*
Poczułem nagle wokół siebie nienawistną woń\\*
Woń, która tłumi wszelki spokój, zrywa wszystkie sny\\*
Z daleka ktoś gdzieś krzyknął nagle krótki rozkaz - goń!\\*
I z czterech stron wypadły na nas cztery gończe psy!

\begin{chorus}
\chorusfirstwords{Obława! Obława! Na młode wilki obława}!\\*
Te dzikie, zapalczywe, w gęstym lesie wychowane!\\*
Krąg śniegu wydeptany! W tym kręgu plama krwawa!\\*
Ciała wilcze kłami gończych psów szarpane!
\end{chorus}

Ten, który na mnie rzucił się, niewiele szczęścia miał\\*
Bo wpadł prosto mi na kły i krew trysnęła z rany\\*
Gdym teraz - ile w łapach sił - przed siebie prosto gnał\\*
Ujrzałem małe wilczki dwa na strzępy rozszarpane!\\*
Zginęły ślepe, ufne tak, puszyste kłębki dwa\\*
Bezradne na tym świecie złym, nie wiedząc kto je zdławił\\*
I zginie także stary wilk, choć życie dobrze zna\\*
Bo z trzema na raz walczy psami i trzech ran na raz krwawi.

\chorusref

Wypadłem na otwartą przestrzeń, pianę z pyska tocząc,\\*
Lecz tutaj także ze wszech stron - zła mnie otacza woń!\\*
A myśliwemu co mnie dojrzał już się śmieją oczy\\*
O ręka pewna, niezawodna podnosi w górę broń!\\*
Rzucam się w bok, na oślep gnam, aż ziemia spod łap tryska\\*
I wtedy pada pierwszy strzał, co kark mi rozszarpuje\\*
Wciąż pędzę słyszę jak on klnie i krew mi płynie z pyska\\*
On strzela po raz drugi! Lecz teraz już pudłuje!

\chorusref

Wyrwałem się z obławy tej, schowałem w jakiś las,\\*
Lecz ile szczęścia miałem w tym to każdy chyba przyzna\\*
Leżałem w śniegu, jak nieżywy długi, długi czas\\*
Po strzale zaś na zawsze mi została krwawa blizna!\\*
Lecz nie skończyła się obława i nie śpią gończe psy\\*
I giną ciągle wilki młode na całym wielkim świecie\\*
Nie dajcie z siebie zedrzeć skór! Brońcie się i wy!\\*
O bracia wilcy! Brońcie się nim wszyscy wyginiecie!

\chorusref
\end{lyrics}



\song[lyricsyear={sierpień 1944},musicyear={sierpień 1944},music={Jan K. Markowski},lyrics={Mirosław Jezierski}]
{Marsz Mokotowa}
\begin{info}Jedna z najbardziej znanych piosenek powstańczych. Śpiewana była również przez partyzantów z oddziałów walczących na Kielecczyźnie i w Krakowskiem. Słowa napisał kpr. pchor. „Karnisz” – Mirosław Jezierski, żołnierz 2 plutonu WSOP (Wewnętrzna Służba Ochrony Powstania) na kwaterze oddziału przy ul. Goszczyńskiego na Mokotowie. Jednostka ta wchodziła w skład pułku Armii Krajowej „Baszta”. Twórcą muzyki był ppor. „Krzysztof” – Jan Markowski.\end{info}

\begin{lyrics}[longestline={Ten pierwszy marsz niech dzień po dniu,}]

\firstwords{Nie grają nam surmy bojowe}\\*
I werble do szturmu nie warczą,\\*
Nam przecież te noce sierpniowe\\*
I prężne ramiona wystarczą.\\*
Niech płynie piosenka z barykad\\*
Wśród bloków, zaułków, ogrodów,\\*
Z chłopcami niech idzie na wypad,\\*
Pod rękę, przez cały Mokotów.

Ten pierwszy marsz ma dziwną moc,\\*
Tak w piersiach gra, aż braknie tchu,\\*
Czy słońca żar, czy chłodna noc,\\*
Prowadzi nas pod ogniem z luf.\\*
Ten pierwszy marsz to właśnie zew,\\*
Niech brzmi i trwa przy huku dział,\\*
Batalion gdzieś rozpoczął szturm,\\*
Spłynęła łza i pierwszy strzał!

Niech wiatr ją poniesie do miasta,\\*
Jak żagiew płonącą i krwawą,\\*
Niech w górze zawiśnie na gwiazdach,\\*
Czy słyszysz, płonąca Warszawo?\\*
Niech zabrzmi w uliczkach znajomych,\\*
W Alejach, gdzie bzy już nie kwitną,\\*
Gdzie w twierdze zmieniły się domy,\\*
A serca z zapału nie stygną!

Ten pierwszy marsz ma dziwną moc,\\*
Tak w piersiach gra, aż braknie tchu,\\*
Czy słońca żar, czy chłodna noc,\\*
Prowadzi nas pod ogniem z luf.\\*
Ten pierwszy marsz niech dzień po dniu,\\*
W poszumie drzew i w sercach drży,\\*
Bez próżnych skarg i zbędnych słów,\\*
To nasza krew i czyjeś łzy!
\end{lyrics}



\song[lyricsyear={sierpień 1914},music={\footnote{na melodię: \textit{Siwa gąska, siwa, po Dunaju pływa}}},lyrics={Wacław Łęcki, Tadeusz Ostrowski}]
{Pierwsza kadrowa}
\begin{info}Piosenka ta powstała w czasie kilkudniowego marszu I Kompani Kadrowej, w dniach 6–12 sierpnia 1914\end{info}

\begin{lyrics}[longestline={Ale przecież dojdziem, byleby iść w nogę.}]

\firstwords{Raduje się serce, raduje się dusza},\\*
Gdy pierwsza kadrowa na wojenkę rusza.

\begin{chorus}
\chorusfirstwords{Oj da, oj da dana, kompanio kochana},\\*
Nie masz to jak pierwsza, nie!
\end{chorus}

Chociaż do Warszawy mamy długą drogę,\\*
Ale przecież dojdziem, byleby iść w nogę.

\chorusref

Kiedy Moskal zdrajca drogę nam zastąpi,\\*
To kul z manlichera nikt mu nie poskąpi.

\chorusref

A gdyby on jeszcze śmiał udawać zucha,\\*
Każdy z nas bagnetem trafi mu do brzucha.

\chorusref

A gdy się szczęśliwie zakończy powstanie,\\*
To pierwsza kadrowa gwardyją zostanie.

\chorusref

A więc piersi naprzód, podniesiona głowa,\\*
Bośmy przecie pierwsza kompania kadrowa

\chorusref
\end{lyrics}



\song[lyricsyear={1917},alt={Marsz Pierwszej Brygady},lyrics={Tadeusz Biernacki, Andrzej T. Hałaciński}]
{Pierwsza Brygada}

\begin{lyrics}[longestline={Na stos rzuciliśmy -- swój życia los,}]

\firstwords{Legiony to -- żołnierska nuta},\\*
Legiony to -- straceńców los,\\*
Legiony to -- rycerska buta,\\*
Legiony to -- ofiarny stos!

\begin{chorus}
\chorusfirstwords{My, Pierwsza Brygada},\\*
Strzelecka gromada,\\*
Na stos rzuciliśmy -- swój życia los,\\*
Na stos, na stos!
\end{chorus}

O, ile mąk, ile cierpienia,\\*
O, ile krwi, wylanych łez,\\*
Pomimo to -- nie ma zwątpienia,\\*
Dodawał sił -- wędrówki kres!

\chorusref

Krzyczeli, żeśmy stumanieni,\\*
Nie wierząc nam, że chcieć -- to móc!\\*
Laliśmy krew osamotnieni,\\*
A z nami był nasz drogi Wódz!

\chorusref

Inaczej się dziś zapatrują\\*
I trafić chcą do naszych dusz.\\*
I mówią, że już nas szanują,\\*
Lecz właśnie czas odwetu już!

\chorusref

Nie chcemy już od was uznania,\\*
Nie waszych mów, ni waszych łez!\\*
Już skończył się czas kołatania\\*
Do waszych serc -- do waszych kies!

\chorusref

Dziś nadszedł czas pokwitowania\\*
Za mękę serc i katusz dni.\\*
Nie chciejcie więc -- politowania,\\*
Zasadą jest: za krew -- chciej krwi!

\chorusref

Dzisiaj już my jednością silni\\*
Tworzymy Polskę -- przodków mit,\\*
Że wy w tej pracy nie dość pilni,\\*
Zostanie wam potomnych wstyd!

\chorusref
Umieliśmy w ogień zapału\\*
Młodzieńczych wiar rozniecić skry,\\*
Nieść życie swe dla ideału\\*
I swoją krew, i marzeń sny.

\chorusref

Potrafim dziś dla potomności\\*
Ostatki swych poświęcić dni.\\*
Wśród fałszów siać siew szlachetności\\*
Miazgą swych ciał, żarem swej krwi.

\chorusref
\end{lyrics}



\song[lyricsyear={1980},music={Natan Tenenbaum},lyrics={Przemysław Gintrowski}]
{Modlitwa o wschodzie słońca}
\begin{info}Napisana w 1980 roku i śpiewana podczas strajków, manifestacji i koncertów "Solidarności" jako swego rodzaju credo tego ruchu, deklaracji. \end{info}

\begin{lyrics}[longestline={Lecz chroń mnie, Panie, od pogardy}]

\firstwords{Każdy Twój wyrok przyjmę twardy}\\*
Przed mocą Twą się ukorzę\\*
Lecz chroń mnie, Panie, od pogardy\\*
Od nienawiści strzeż mnie, Boże

Wszak Ty jesteś niezmierzone dobro\\*
Którego nie wyrażą słowa

Więc mnie od nienawiści obroń\\*
I od pogardy mnie zachowaj

Co postanowisz, niech się ziści\\*
Niechaj się wola Twoja stanie\\*
Ale zbaw mnie od nienawiści\\*
Ocal mnie od pogardy, Panie
\end{lyrics}



\song[lyricsyear={1978},music={do melodii piosenki L'Estaca (Słup) katalońskiego pieśniarza Lluísa Llacha},lyrics={Jacek Kaczmarski}]
{Mury}

\begin{lyrics}[longestline={Śpiewali więc, klaskali w rytm, jak wystrzał poklask ich brzmiał,}]

\firstwords{On natchniony i młody był, ich nie policzyłby nikt}\\*
On im dodawał pieśnią sił, śpiewał że blisko już świt.\\*
Świec tysiące palili mu, znad głów podnosił się dym,\\*
Śpiewał, że czas by runął mur...\\*
Oni śpiewali wraz z nim:

\begin{chorus}
\chorusfirstwords{Wyrwij murom zęby krat}!\\*
Zerwij kajdany, połam bat!\\*
A mury runą, runą, runą\\*
I pogrzebią stary świat!
\end{chorus}

Wkrótce na pamięć znali pieśń i sama melodia bez słów\\*
Niosła ze sobą starą treść, dreszcze na wskroś serc i głów.\\*
Śpiewali więc, klaskali w rytm, jak wystrzał poklask ich brzmiał,\\*
I ciążył łańcuch, zwlekał świt...\\*
On wciąż śpiewał i grał:

\chorusref

Aż zobaczyli ilu ich, poczuli siłę i czas,\\*
I z pieśnią, że już blisko świt szli ulicami miast;\\*
Zwalali pomniki i rwali bruk - Ten z nami! Ten przeciw nam!\\*
Kto sam ten nasz najgorszy wróg!\\*
A śpiewak także był sam.

Patrzył na równy tłumów marsz,\\*
Milczał wsłuchany w kroków huk,\\*
A mury rosły, rosły, rosły\\*
Łańcuch kołysał się u nóg...

Patrzy na równy tłumów marsz,\\*
Milczy wsłuchany w kroków huk,\\*
A mury rosną, rosną, rosną\\*
Łańcuch kołysze się u nóg...
\end{lyrics}



\song[lyricsyear={7.5.83/3.6.87},lyrics={Jacek Kaczmarski}]
{Nasza klasa}

\begin{lyrics}[longestline={Gdy wśród tych - nieobcych - twarzy}]

\firstwords{Co się stało z naszą klasą}\\*
Pyta Adam w Tel-Avivie,\\*
Ciężko sprostać takim czasom,\\*
Ciężko w ogóle żyć uczciwie -\\*
Co się stało z naszą klasą?\\*
Wojtek w Szwecji, w porno klubie\\*
Pisze - dobrze mi tu płacą\\*
\markverse[marktext={bis}]{Za to, co i tak wszak lubię.}

Kaśka z Piotrkiem są w Kanadzie,\\*
Bo tam mają perspektywy,\\*
Staszek w Stanach sobie radzi,\\*
Paweł do Paryża przywykł,\\*
Gośka z Przemkiem ledwie przędą,\\*
W maju będzie trzeci bachor,\\*
Próżno skarżą się urzędom,\\*
\markverse[marktext={bis}]{Że też chcieli by na zachód,}

Za to Magda jest w Madrycie\\*
I wychodzi za Hiszpana,\\*
Maciek w grudniu stracił życie,\\*
Gdy chodzili po mieszkaniach,\\*
Janusz, ten, co zawiść budził,\\*
Że go każda fala niesie,\\*
Jest chirurgiem, leczy ludzi,\\*
\markverse[marktext={bis}]{Ale brat mu się powiesił,}

Marek siedzi za odmowę,\\*
Bo nie strzelał do Michała,\\*
A ja piszę ich historię\\*
I to już jest klasa cała.\\*
Jeszcze Filip, fizyk w Moskwie -\\*
Dziś nagrody różne zbiera,\\*
Jeździ, kiedy chce do Polski,\\*
\markverse[marktext={bis}]{Był przyjęty przez premiera.}

Odnalazłem klasę całą -\\*
Na wygnaniu, w kraju, w grobie,\\*
Ale coś się pozmieniało,\\*
Każdy sobie żywot skrobie -\\*
Odnalazłem całą klasę\\*
Wyrośniętą i dojrzałą,\\*
Rozdrapałem młodość naszą,\\*
\markverse[marktext={bis}]{Lecz za bardzo nie bolało...}

Już nie chłopcy, lecz mężczyźni,\\*
Już kobiety - nie dziewczyny.\\*
Młodość szybko się zabliźni,\\*
Nie ma w tym niczyjej winy;\\*
Wszyscy są odpowiedzialni,\\*
Wszyscy mają w życiu cele,\\*
Wszyscy w miarę są - normalni,\\*
\markverse[marktext={bis}]{Ale przecież - to niewiele...}

Nie wiem sam, co mi się marzy,\\*
Jaka z gwiazd nade mną świeci,\\*
Gdy wśród tych - nieobcych - twarzy\\*
Szukam ciągle twarzy - dzieci,\\*
Czemu wciąż przez ramię zerkam,\\*
Choć nie woła nikt - kolego!\\*
Że ktoś ze mną zagra w berka,\\*
\markverse[marktext={bis}]{Lub przynajmniej w chowanego...}

Własne pędy, własne liście,\\*
Zapuszczamy - każdy sobie\\*
I korzenie oczywiście\\*
Na wygnaniu, w kraju, w grobie,\\*
W dół, na boki, wzwyż ku słońcu,\\*
Na stracenie, w prawo - w lewo...\\*
Kto pamięta, że to w końcu\\*
\markverse[marktext={bis}]{Jedno i - to samo drzewo...}
\end{lyrics}



\song[lyricsyear={1925},alt={Marynarka wojenna},musicyear={1925},music={Adam Kowalski},lyrics={Adam Kowalski}]
{Morze, nasze morze}
\begin{info}Hymn Marynarki Wojennej.\end{info}

\begin{lyrics}[longestline={albo na dnie z honorem lec, z honorem lec.}]

\firstwords{Chociaż każdy z nas jest młody},\\*
lecz go starym wilkiem zwą.\\*
My, strażnicy polskiej wody,\\*
marynarze polscy to.

\begin{chorus}
\chorusfirstwords{Morze, nasze morze},\\*
wiernie ciebie będziem strzec.\\*
Mamy rozkaz cię utrzymać,\\*
albo na dnie, na dnie twoim lec,\\*
albo na dnie z honorem lec, z honorem lec.
\end{chorus}

Żadna siła, żadna burza,\\*
nie odbierze Gdańska nam.\\*
Nasza flota, choć nieduża,\\*
wiernie strzeże portu bram.

\chorusref

Morze, nasze morze...
\end{lyrics}



\song[lyricsyear={4 sierpnia 1944},music={melodia Hymnu Podhalańskiego},lyrics={Józef "Ziutek" Szczepański}]
{Pałacyk Michla}
\begin{info} Wojenny hymn harcerskiego Batalionu Parasol. Piosenka zdobyła sobie ogromną popularność, gdyż rozpowszechniał ją wśród powstańców Warszawy, zwłaszcza w Śródmieściu, Mieczysław Fogg.\end{info}

\begin{lyrics}[longestline={to jest nasz „Miecio” w kółko golony — hej!}]

\firstwords{Pałacyk Michla, Żytnia, Wola},\\*
bronią jej chłopcy od „Parasola”,\\*
choć na „tygrysy” mają visy —\\*
to warszawiaki, fajne chłopaki — są!

\begin{chorus}
\chorusfirstwords{Czuwaj wiaro i wytężaj słuch},\\*
pręż swój młody duch, pracując za dwóch!\\*
Czuwaj wiaro i wytężaj słuch,\\*
pręż swój młody duch, jak stal!
\end{chorus}

Każdy chłopaczek chce być ranny...\\*
sanitariuszki — morowe panny,\\*
i gdy cię kula trafi jaka,\\*
poprosisz pannę — da ci buziaka — hej!

\chorusref

Z tyłu za linią dekowniki,\\*
intendentura, różne umrzyki,\\*
gotują zupę, czarną kawę —\\*
i tym sposobem walczą za sprawę — hej!

\chorusref

Za to dowództwo jest morowe,\\*
bo w pierwszej linii nadstawia głowę,\\*
a najmorowszy z przełożonych,\\*
to jest nasz „Miecio” w kółko golony — hej!

\chorusref

Wiara się bije, wiara śpiewa,\\*
szkopy się złoszczą, krew ich zalewa,\\*
różnych sposobów się imają,\\*
co chwila „szafę” nam posuwają — hej!

\chorusref

Lecz na nic „szafa” i granaty,\\*
za każdym razem dostają baty\\*
i co dzień się przybliża chwila,\\*
że zwyciężymy! I do cywila — hej!

\chorusref
\end{lyrics}



\song[lyricsyear={ok. 1918},music={Leon Łuskino},lyrics={Bolesław Lubicz-Zahorski, Leon Łuskino}]
{Piechota}

\begin{lyrics}[longestline={Nie noszą lampasów, lecz szary ich strój,}]

\firstwords{Nie noszą lampasów, lecz szary ich strój},\\*
Nie noszą ni srebra, ni złota.\\*
\begin{markverses}[marktext={x2}]%
Lecz w pierwszym szeregu podąża na bój\\*
Piechota, ta szara piechota.
\end{markverses}

\begin{chorus}
\chorusfirstwords{Maszerują chłopcy, maszerują},\\*
karabiny błyszczą, szary strój,\\*
a przed nimi drzewa salutują,\\*
bo za naszą Polskę idą w bój!
\end{chorus}

Idą, a w słońcu kołysze się stal.\\*
Dziewczęta zerkają zza plota,\\*
\begin{markverses}[marktext={x2}]%
A oczy ich dumne utkwione są w dal,\\*
Piechota, ta szara piechota!
\end{markverses}

\chorusref

Nie grają im surmy, nie huczy im róg,\\*
A śmierć im pod stopy się miota,\\*
\begin{markverses}[marktext={x2}]%
Lecz w pierwszym szeregu podąża na bój\\*
Piechota, ta szara piechota.
\end{markverses}

\chorusref
\end{lyrics}



\song[lyricsyear={1918-1919},lyrics={Jerzy Braun}]
{Płonie ognisko}
\begin{info}Jedna z najpopularniejszych pieśni harcerskich.Twórca pieśni, Jerzy Braun, w momencie pisania utworu był maturzystą II Gimnazjum w Tarnowie. Drużynowy, o którym mówią słowa, to ,,Druh Bajdała'' – Władysław Wodniecki, kierujący drużyną harcerską, do której należał autor.\end{info}

\begin{lyrics}[longestline={O obrońcach naszych polskich granic,}]

\firstwords{Płonie ognisko i szumią knieje},\\*
Drużynowy jest wśród nas.\\*
Opowiada starodawne dzieje,\\*
Bohaterski wskrzesza czas.\\*
\begin{indented}
O rycerstwie spod kresowych stanic,\\*
O obrońcach naszych polskich granic,\\*
A ponad nami wiatr szumi, wieje\\*
I dębowy huczy las.
\end{indented}

Już do powrotu głos trąbki wzywa,\\*
Alarmują ze wszystkich stron!\\*
Staje wiara w ordynku szczęśliwa,\\*
Serca bija w zgodny ton!\\*
Każda twarz się uniesieniem płoni,\\*
Każdy laskę krzepko dzierży w dłoni\\*
I z młodzieńczej się piersi wyrywa\\*
Pieśń potężna, pieśń jak dzwon.
\end{lyrics}



\song
{Płynie Wisła, płynie}
\begin{info}Pieśń ta jest anonimową odmianą słowną i melodyczną napisanego przez E. Wasilewskiego w 1840 r. krakowiaka "Od południa stoi". Odegrała ona wielką rolę wychowawczą w czasach niewoli, okupacji i zniewolenia komunistycznego.\end{info}

\begin{lyrics}[multicol=true, longestline={''Ojcze nasz'' i ''Zdrowaś''}]

\firstwords{Płynie Wisła, płynie}\\*
\markverse[marktext={bis}]{Po polskiej krainie,}\\*
\begin{markverses}[marktext={bis}]%
Zobaczyła Kraków,\\*
pewnie go nie minie.
\end{markverses}

Zobaczyła Kraków,\\*
\markverse[marktext={bis}]{Wnet go pokochała,}\\*
\begin{markverses}[marktext={bis}]%
A w dowód miłości\\*
wstęgą opasała.
\end{markverses}

Chociaż się schowała\\*
\markverse[marktext={bis}]{W Niepołomskie lasy,}\\*
\begin{markverses}[marktext={bis}]%
I do morza wpada,\\*
płynie jak przed czasy.
\end{markverses}

Nad moją kolebą\\*
\markverse[marktext={bis}]{Matka się schylała,}\\*
\begin{markverses}[marktext={bis}]%
I po polsku pacierz\\*
mówić nauczała.
\end{markverses}

''Ojcze nasz'' i ''Zdrowaś''\\*
\markverse[marktext={bis}]{I ''Skład Apostolski'',}\\*
\begin{markverses}[marktext={bis}]%
Bym do samej śmierci\\*
kochał naród polski.
\end{markverses}

Bo ten naród polski\\*
\markverse[marktext={bis}]{Ma ten urok w sobie,}\\*
\begin{markverses}[marktext={bis}]%
Kto go raz pokochał,\\*
nie zapomni w grobie.
\end{markverses}

Abym gdy dorosnę\\*
\markverse[marktext={bis}]{Wziął Polkę za żonę}\\*
\begin{markverses}[marktext={bis}]%
Bo tylko Polakom\\*
Laszki przeznaczone.
\end{markverses}

Niech Francuz Francuzkę\\*
\markverse[marktext={bis}]{Niemiec kocha Niemkę}\\*
\begin{markverses}[marktext={bis}]%
Ja zaś wolę Polkę,\\*
niźli cudzoziemkę.
\end{markverses}

I to wszystko razem\\*
\markverse[marktext={bis}]{Od matki słyszałem}\\*
\begin{markverses}[marktext={bis}]%
Czego nie zapomnę\\*
jak nie zapomniałem.
\end{markverses}

Płynie Wisła płynie,\\*
\markverse[marktext={bis}]{Po polskiej krainie}\\*
\begin{markverses}[marktext={bis}]%
A dopóki płynie\\*
Polska nie zaginie.
\end{markverses}
\end{lyrics}



\song[lyricsyear={luty 1863},alt={Pieśń z obozu Jeziorańskiego},music={Alfred Bojarski},lyrics={Wincenty Pol}]
{Sygnał}
\begin{info}Pieśń powstała jako akt pożegnania mężczyzn odchodzących do partyzantki. Początkowo utwór wykonywali tylko powstańcy w obozie Antoniego Jeziorańskiego. Szybko jednak zdobył powszechną popularność w oddziałach na terenie wszystkich zaborów\end{info}

\begin{lyrics}[longestline={W krwawym polu srebrne ptaszę,}]

\firstwords{W krwawym polu srebrne ptaszę},\\*
Poszli w boje chłopcy nasze.

\begin{chorus}
\chorusfirstwords{Hu, ha! Krew gra}!\\*
Duch gra! Hu, ha\\*
Niechaj Polska zna,\\*
Jakich synów ma.
\end{chorus}

Obok Orła znak Pogoni,\\*
Poszli nasi w bój bez broni.

\begin{chorus}
\chorusfirstwords{Hu, ha! Krew gra}!\\*
Duch gra! Hu,ha!\\*
Matko-Polsko żyj!\\*
Jezus, Maria, bij!
\end{chorus}

Naszym braciom dopomagaj,\\*
Nieprzyjaciół naszych smagaj.

\begin{chorus}
\chorusfirstwords{Hu, ha! Wiatr gra}!\\*
Krew gra! Wiatr gra!\\*
Niechaj Polska zna,\\*
Jakich synów ma!
\end{chorus}
\end{lyrics}
