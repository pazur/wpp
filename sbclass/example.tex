\documentclass[12pt,a4paper,twoside]{songbook}

\usepackage{t1enc}
\usepackage[polish]{babel}
\usepackage[utf8]{inputenc}

\usepackage[width=15cm]{geometry}

\usepackage{layout} % DEBUGGING, TESTING
\usepackage{lipsum} % DEBUGGING, TESTING
\usepackage{tikz}


\makeindex{fwidx}
\makeindex{titidx}

\title{Śpiewnik}
\date{23 września 2012}
\author{Magda \and Tomek \and Maciek}

\begin{document}
\maketitle
\setlipsumdefault{1}





\begin{lyrics}
  
  Maciek \\
  Maciek \\
  Maciek \\
  Maciek
  
  The first line. \\
  \vin \flagverse{Ref.} The second line. \tikzmark{2nd} \\
  A very long third line that goes on. \\
  A fourth line. \\
  And finally a last line. \tikzmark{4th}\\
  maffa
  \hspace{\versewidth}\tikzmark{righta}

  The first line. \\
  \flagverse{Ref.} The second line. \\
  A very long third line that goes on. \\
  \tikzmark{2nd}A fourth line. \\
  And finally a last line. \tikzmark{4th}
  \hspace{\versewidth}\tikzmark{rightb}
  \begin{tikzpicture}[overlay, remember picture]
    \draw [decoration={brace,amplitude=0.5em},decorate,gray]
    ($(righta)!(2nd.north)!($(righta)-(0,1)$)$) -- node[black, right=0.7em] {x2}  ($(righta)!(4th.south)!($(righta)-(0,1)$)$);
  \end{tikzpicture}

  The first line. \\
  \flagverse{Ref.} line. \\
  A very long third \\
  \tikzmark{m2nd}A fourth line. \\
  And finally line. \tikzmark{m4th}
  \hspace{\versewidth}\tikzmark{rightb}
  \begin{tikzpicture}[overlay, remember picture]
    \draw [decoration={brace,amplitude=0.5em},decorate,gray]
    ($(righta)!(m2nd.north)!($(righta)-(0,1)$)$) -- node[black, right=0.7em] {x2}  ($(righta)!(m4th.south)!($(righta)-(0,1)$)$);
  \end{tikzpicture}
  
  Maciek \\ 
  Maciek \\
  Maciek
\end{lyrics}








\begin{lyrics}[longestline={A very long third line that goes on.}]
\begin{tikzpicture}
  \matrix (m) [matrix of nodes,column 1/.style={anchor=west}]
    {  
    The first line. \\
    The second line. \\
    A very long third line that goes on. \\
    A fourth line. \\
    And finally a last line. \\
    maffa \\
    
    \vspace{1in}
    
    The first line. \\
    The second line. \\
    A very long third line that goes on. \\
    A fourth line. \\
    And finally a last line. \\
    };

    \draw [decoration={brace,amplitude=0.7em},decorate,gray]
        (m-2-1.north -| m.east) -- node[black, right=0.7em] {x2} (m-3-1.south -| m.east);
\end{tikzpicture}

\begin{tikzpicture}
  \matrix (m) [matrix of nodes,column 1/.style={anchor=west}]
    {  
    The first line. \\
    The second line. \\
    A very long third line that goes on. \\
    A fourth line. \\
    And finally a last line. \\
    };

    \draw [decoration={brace,amplitude=0.7em},decorate,gray]
        (m-2-1.north -| m.east) -- node[black, right=0.7em] {x2} (m-3-1.south -| m.east);
\end{tikzpicture}


\end{lyrics}

\song[alt=Pieśń Legionów Polskich we Włoszech,
      lyrics=Józef Wybicki,
      lyricsyear=1797]
      {Mazurek Dąbrowskiego}

  \begin{info}
    Pierwotnie hymn, jako Pieśń Legionów Polskich we Włoszech, został
    napisany przez Józefa Wybickiego. Autor melodii opartej na motywach
    ludowego mazurka (właściwie mazura) jest nieznany.

    Pieśń powstała w~dniach 16-19 lipca 1797 we włoskim miasteczku
    Reggio nell'Emilia w Republice Cisalpińskiej (w dzisiejszych
    Włoszech). Pierwszy raz została wykonana publicznie 20 lipca 1797
    roku[3]. Tekst ogłoszono po raz pierwszy w Mantui w lutym 1799
    w~gazetce "Dekada Legionowa".
  \end{info}
  
  \begin{lyrics}
    \firstwords{Jeszcze Polska nie zginęła},\\
    Kiedy my żyjemy.\\
    Co nam obca przemoc wzięła,\\
    Szablą odbierzemy.

    
    \flagverse{Ref.}Marsz, marsz, Dąbrowski,\\
    \begin{bis}%
      Z ziemi włoskiej do Polski.\\
      Za twoim przewodem\\
      Złączym się z narodem.%
    \end{bis}
    
    Przejdziem Wisłę, przejdziem Wartę,\\
    Będziem Polakami.\\
    Dał nam przykład Bonaparte,\\
    Jak zwyciężać mamy.
      
    Marsz, marsz, Dąbrowski...

    Jak Czarniecki do Poznania\\
    Po szwedzkim zaborze,\\
    Dla ojczyzny ratowania\\
    Wrócim się przez morze.
    
    Marsz, marsz, Dąbrowski...

    Już tam ojciec do swej Basi\\
    Mówi zapłakany —\\
    Słuchaj jeno, pono nasi\\
    Biją w tarabany.

    Marsz, marsz, Dąbrowski...
  \end{lyrics}


\song{Mazurek Dąbrowskiego}
  \begin{info}
    \lipsum
  \end{info}

  
  \begin{lyrics}[multicol=true, longestline=Przejdziem Wisłę przejdziem Wartę]
    \firstwords{Jeszcze Polska nie umarła},\\
    kiedy my żyjemy.\\
    Co nam obca moc wydarła,\\
    szablą odbijemy.
    
    \flagverse{Ref.}Marsz, marsz, Dąbrowski\\
    do Polski z ziemi włoski\\
    za Twoim przewodem\\
    złączem się z narodem.
    
    Jak Czarnecki do Poznania\\
    wracał się przez morze\\
    dla ojczyzny ratowania\\
    po szwedzkim rozbiorze.
    
    Marsz, marsz, Dąbrowski...
    
    
    Przejdziem Wisłę przejdziem Wartę\\*
    będziem Polakami\\*
    dał nam przykład Bonaparte\\*
    jak zwyciężać mamy.\\*
    
    Marsz, marsz, Dąbrowski...
    
    Niemiec, Moskal nie osiędzie,\\
    gdy jąwszy pałasza,\\
    hasłem wszystkich zgoda będzie\\
    i ojczyzna nasza.

    Marsz, marsz, Dąbrowski...

    Już tam ojciec do swej Basi\\
    mówi zapłakany:\\
    ,,słuchaj jeno, pono nasi\\
    biją w tarabany.''

    Marsz, marsz, Dąbrowski...

    Na to wszystkich jedne głosy:\\
    ,,Dosyć tej niewoli\\
    mamy Racławickie Kosy,\\
    Kościuszkę, Bóg pozwoli.''
  \end{lyrics}

  
\song[lyrics=Mieczysław Kozar-Słobódzki,
      lyricsyear= ok. 1914,
      music=Kazimierz Wroczyński
]{Białe róże}

  \begin{info}
    \lipsum[1]
  \end{info}
  \setlength{\vgap}{10pt}
  \begin{lyrics}
    \firstwords{Rozkwitały pąki białych róż}.\\
    Wróć Jasieńku, z tej wojenki już!\\
    \vin Wróć! ucałuj jak za dawnych lat,\\
    \vin Dam ci za to róży najpiękniejszy kwiat! \emph{(x2)}
    %Wróć! ucałuj jak za dawnych lat,\\
    %Dam ci za to róży najpiękniejszy kwiat!
    
    Przekwitały pąki białych róż.\\
    Przeszło lato, jesień, zima już.\\
    Cóż ci teraz dam Jasieńku hej,\\
    Gdy z wojenki wrócisz do dziewczyny twej?\\
    
    Cóż ci teraz dam Jasieńku hej,\\
    Gdy z wojenki wrócisz do dziewczyny twej?
    
    Jasieńkowi nic nie potrza już,\\
    Bo mu kwitną nowe pąki róż,\\
    Tam nad jarem, gdzie w wojence padł\\
    Rozkwitł u mogiły białej róży kwiat.\\
    Tam nad jarem, gdzie w wojence padł\\
    Rozkwitł u mogiły białej róży kwiat.\\
  \end{lyrics}

\song{Pierwsza brygada}

  \begin{info}
    \lipsum[1]
  \end{info}

  \begin{lyrics}[longestline=Legiony to straceńców los]
  \firstwords{Legiony to żołnierska nuta}, \\
  Legiony to straceńców los, \\
  Legiony to żołnierska buta, \\
  Legiony to ofiarny stos
  
  My, Pierwsza Brygada,\\
  Strzelecka gromada,\\
  Na stos rzuciliśmy\\
  Swój życia los,\\
  Na stos, na stos!
  \end{lyrics}



 \firstwords{Suspendisse potenti}
\firstwords{Duis ullamcorper}
\firstwords{euismod turpis}
\firstwords{sit amet}
\firstwords{tristique Fusce}
\firstwords{ac elit}
\firstwords{eu leo}
\firstwords{auctor hendrerit}
\firstwords{Mauris a}
\firstwords{odio lacus}
\firstwords{Fusce tincidunt}
\firstwords{libero non}
\firstwords{dolor luctus}
\firstwords{nec commodo}
\firstwords{enim mollis}
\firstwords{Class aptent}
\firstwords{taciti sociosqu}
\firstwords{ad litora}
\firstwords{torquent per}
\firstwords{conubia nostra}
\firstwords{per inceptos}
\firstwords{himenaeos Proin}
\firstwords{elementum aliquet}
\firstwords{leo Quisque}
\firstwords{vulputate enim}
\firstwords{non ante}
\firstwords{elementum euismod}
\firstwords{Praesent nisl}
\firstwords{quam facilisis}
\firstwords{et sodales}
\firstwords{blandit pellentesque}
\firstwords{non risus}
\firstwords{Nulla facilisi}
\firstwords{Maecenas at}
\firstwords{dui quam}
\firstwords{eu imperdiet}
\firstwords{nunc Praesent}
\firstwords{mauris erat}
\firstwords{pretium a}
\firstwords{lacinia hendrerit}
\firstwords{porttitor ut}
\firstwords{arcu Vestibulum}
\firstwords{ante ipsum}
\firstwords{primis in}
\firstwords{faucibus orci}
\firstwords{luctus et}
\firstwords{ultrices posuere}
\firstwords{cubilia Curae}
\firstwords{Class aptent}
\firstwords{taciti sociosqu}
\firstwords{ad litora}
\firstwords{torquent per}
\firstwords{conubia nostra}
\firstwords{per inceptos}
\firstwords{himenaeos Integer}
\firstwords{nulla arcu}
\firstwords{tristique in}
\firstwords{pellentesque nec}
\firstwords{luctus ut}
\firstwords{lacus}
\firstwords{}
\firstwords{Maecenas}
\firstwords{tortor nulla}
\firstwords{fermentum nec}
\firstwords{lacinia at}
\firstwords{venenatis at}
\firstwords{nisi Vivamus}
\firstwords{ac commodo}
\firstwords{sem Mauris}
\firstwords{eu ligula}
\firstwords{sed tortor}
\firstwords{sollicitudin hendrerit}
\firstwords{ut id}
\firstwords{sem Aenean}
\firstwords{ultricies fermentum}
\firstwords{facilisis Nulla}
\firstwords{facilisi In}
\firstwords{scelerisque eros}
\firstwords{et leo}
\firstwords{pellentesque laoreet}
\firstwords{Morbi tempus}
\firstwords{dignissim convallis}
\firstwords{Quisque ornare}
\firstwords{turpis at}
\firstwords{convallis mollis}
\firstwords{diam lorem}
\firstwords{tincidunt turpis}
\firstwords{a malesuada}
\firstwords{purus nisi}
\firstwords{nec arcu}
\firstwords{Duis mollis}
\firstwords{sem ac}
\firstwords{velit auctor}
\firstwords{convallis Suspendisse}
\firstwords{accumsan cursus}
\firstwords{aliquet Fusce}
\firstwords{suscipit hendrerit}
\firstwords{augue sit}
\firstwords{amet vulputate}
\firstwords{Vivamus urna}
\firstwords{mauris sagittis}
\firstwords{vehicula sodales}
\firstwords{at laoreet}
\firstwords{vitae dolor}
\firstwords{Praesent interdum}
\firstwords{dignissim diam}
\firstwords{nec porta}
\firstwords{felis blandit}
\firstwords{in Cum}
\firstwords{sociis natoque}
\firstwords{penatibus et}
\firstwords{magnis dis}
\firstwords{parturient montes nascetur ridiculus mus}
\firstwords{Exegi monumentum aere perennius}
\firstwords{Longum titlum longeli titli longus longus longus}

\printindex{titidx}{Piosenki wg tytułów}
\printindex{fwidx}{Piosenki wg pierwszych słów}

\end{document}
