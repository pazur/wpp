\documentclass[a4paper,twoside,noinfo,nofrontmatter]{songbook}

\usepackage{t1enc}
\usepackage[polish]{babel}
\usepackage[utf8]{inputenc}

\usepackage[body={15cm,22cm},headsep=.5in,voffset=.5in]{geometry}

\usepackage{layout} % DEBUGGING, TESTING
\usepackage{lipsum} % DEBUGGING, TESTING

\makeindex{fwidx}
\makeindex{titidx}

\usepackage{graphicx}
\begin{document}

\begin{front}
\begin{titlepage}
\begin{center}
  {\VERYHUGE \bfseries\textsc{Śpiewnik}}

  \vspace{0.2in}
  {\Huge\bfseries na}
  \vspace{0.2in}
  
  {\HUGE\bfseries wieczory pieśni patriotycznej}
\end{center}
\vspace{0.5in}
\begin{center}
  \includegraphics[width=3in]{img/godlo-small.png}
\end{center}
\vspace{0.55in}
\begin{center}
  {\Huge\bfseries 23 września 2012}
\end{center}
\end{titlepage}

\begin{dedication}
  Kochanemu Tacie\\
  z okazji 50. urodzin
\end{dedication}
\end{front}

\song[alt=Pieśń Legionów Polskich we Włoszech\footnote{To oryginalny tytuł},
lyrics={Józef Wybicki\footnote{Blabla}},
      lyricsyear={1797\footnote{Ale tak do końca nie wiadomo}}]
      {Mazurek Dąbrowskiego}

\begin{info}
    Pierwotnie hymn, jako Pieśń Legionów Polskich we Włoszech, został
    napisany przez Józefa Wybickiego. Autor melodii opartej na motywach
    ludowego mazurka (właściwie mazura) jest nieznany.

    Pieśń powstała w~dniach 16-19 lipca 1797 we włoskim miasteczku
    Reggio nell'Emilia w Republice Cisalpińskiej (w dzisiejszych
    Włoszech). Pierwszy raz została wykonana publicznie 20 lipca 1797
    roku[3]. Tekst ogłoszono po raz pierwszy w Mantui w lutym 1799
    w~gazetce "Dekada Legionowa".
\end{info}

  \begin{lyrics}
    \firstwords{Jeszcze Polska nie zginęła},\\
    Kiedy my żyjemy.\\
    \markverse{Co nam obca przemoc wzięła,}\\
    Szablą odbierzemy.
    
    \begin{chorus}%
      \chorusfirstwords{Marsz, marsz, Dąbrowski},\\
      Z ziemi włoskiej do Polski.\\
      Za twoim przewodem\\
      Złączym się z narodem.
    \end{chorus}
    
    \begin{markverses}[marktext=costam]%
    Przejdziem Wisłę, przejdziem Wartę,\\
    Będziem Polakami.\\
    Dał nam przykład Bonaparte,\\
    Jak zwyciężać mamy.
    \end{markverses}
    
    \chorusref
    
    Jak Czarniecki do Poznania\\
    Po szwedzkim zaborze,\\
    Dla ojczyzny ratowania\\
    Wrócim się przez morze.
        
    \chorusref

    Już tam ojciec do swej Basi\\
    Mówi zapłakany --\\
    Słuchaj jeno, pono nasi\\
    Biją w tarabany.

    \chorusref
  \end{lyrics}



\song{Mazurek Dąbrowskiego}
\begin{info}
    \lipsum
\end{info}

  
  \begin{lyrics}[multicol=true, longestline=Przejdziem Wisłę przejdziem Wartę]
    \firstwords{Jeszcze Polska nie umarła},\\
    kiedy my żyjemy.\\
    Co nam obca moc wydarła,\\
    szablą odbijemy.
    
    \begin{chorus}
    \chorusfirstwords{Marsz, marsz, Dąbrowski}\\
    do Polski z ziemi włoski\\
    za Twoim przewodem\\
    złączem się z narodem.
    \end{chorus}%
    
    Jak Czarnecki do Poznania\\
    wracał się przez morze\\
    dla ojczyzny ratowania\\
    po szwedzkim rozbiorze.
    
    \chorusref
    
    Przejdziem Wisłę przejdziem Wartę\\*
    będziem Polakami\\*
    dał nam przykład Bonaparte\\*
    jak zwyciężać mamy.\\*
    
    \chorusref
    
    Niemiec, Moskal nie osiędzie,\\
    gdy jąwszy pałasza,\\
    hasłem wszystkich zgoda będzie\\
    i ojczyzna nasza.

    \chorusref

    Już tam ojciec do swej Basi\\
    mówi zapłakany:\\
    ,,słuchaj jeno, pono nasi\\
    biją w tarabany.''

    \chorusref

    Na to wszystkich jedne głosy:\\
    ,,Dosyć tej niewoli\\
    mamy Racławickie Kosy,\\
    Kościuszkę, Bóg pozwoli.''
  \end{lyrics}
  
\song[lyrics={Mieczysław Kozar-Słobódzki\footnote{Coś tam dłuższego}},
      lyricsyear= ok. 1914,
      music=Kazimierz Wroczyński
]{Białe róże}

\begin{info}
    \lipsum[1]
\end{info}
  
  \begin{lyrics}[longestline={Gdy z wojenki wrócisz do dziewczyny twej?}]
    \firstwords{Rozkwitały pąki białych róż}.\\
    Wróć Jasieńku, z tej wojenki już!\\    
    \begin{markverses}[marktext={powtórzcie to ludzie}]%
      Wróć! ucałuj jak za dawnych lat,\\
      Dam ci za to róży najpiękniejszy kwiat!
    \end{markverses}
    
    \markverse{Przekwitały pąki białych róż.}\\
    \markverse[marktext={i to również!}]{Przeszło lato, jesień, zima już.}\\
    \begin{markverses}%
      Cóż ci teraz dam Jasieńku hej,\\
      Gdy z wojenki wrócisz do dziewczyny twej?
    \end{markverses}
    
    Jasieńkowi nic nie potrza już,\\
    Bo mu kwitną nowe pąki róż,\\
    \begin{markverses}%
      Tam nad jarem, gdzie w wojence padł\\
      Rozkwitł u mogiły białej róży kwiat.
    \end{markverses}
  \end{lyrics}

\song{Pierwsza brygada}

\begin{info}
    \lipsum[1]
\end{info}

  \begin{lyrics}[longestline=Legiony to straceńców los]
  \firstwords{Legiony to żołnierska nuta}, \\
  Legiony to straceńców los, \\
  Legiony to żołnierska buta, \\
  Legiony to ofiarny stos
  
  My, Pierwsza Brygada,\\
  Strzelecka gromada,\\
  Na stos rzuciliśmy\\
  Swój życia los,\\
  Na stos, na stos!
  \end{lyrics}

\printindex{titidx}{Piosenki wg tytułów}
\printindex{fwidx}{Piosenki wg pierwszych słów}

\end{document}
