\song[lyricsyear={1797},alt={Pieśń Legionów Polskich we Włoszech},lyrics={Józef Wybicki}]
{Mazurek Dąbrowskiego}
\begin{info}Pierwotnie hymn, jako Pieśń Legionów Polskich we Włoszech, został napisany przez Józefa Wybickiego. Autor melodii opartej na motywach ludowego mazurka (właściwie mazura) jest nieznany. Pieśń powstała w dniach 16-19 lipca 1797 we włoskim miasteczku Reggio nell'Emilia w Republice Cisalpińskiej (w dzisiejszych Włoszech). Pierwszy raz została wykonana publicznie 20 lipca 1797 roku. Tekst ogłoszono po raz pierwszy w Mantui w lutym 1799 w gazetce "Dekada Legionowa".\end{info}

\begin{lyrics}[longestline={Przejdziem Wisłę, przejdziem Wartę,}]

\firstwords{Jeszcze Polska nie zginęła},\\*
Kiedy my żyjemy.\\*
Co nam obca przemoc wzięła,\\*
Szablą odbierzemy.

\begin{chorus}
\chorusfirstwords{Marsz, marsz, Dąbrowski},\\*
Z ziemi włoskiej do Polski.\\*
Za twoim przewodem\\*
Złączym się z narodem.
\end{chorus}

Przejdziem Wisłę, przejdziem Wartę,\\*
Będziem Polakami.\\*
Dał nam przykład Bonaparte,\\*
Jak zwyciężać mamy.

\chorusref

Jak Czarniecki do Poznania\\*
Po szwedzkim zaborze,\\*
Dla ojczyzny ratowania\\*
Wrócim się przez morze.

\chorusref

Już tam ojciec do swej Basi\\*
Mówi zapłakany --\\*
Słuchaj jeno, pono nasi\\*
Biją w tarabany.

\chorusref
\end{lyrics}



\song[lyricsyear={1942},alt={Kołysanka leśna},lyrics={autor nieznany\footnote{Autorstwo przypisywane bywa Stanisławowi Magierskiemu z Lublina, Bronisławowi Królowi z lwowskiej grupy poetyckiej ,,Żagiew'' oraz Krystynie Krahelskiej.}}]
{Dziś do ciebie przyjść nie mogę}
\begin{info}Piosenka partyzancka z przełomu 1942–1943. W okupowany kraj popłynęła z Lubelszczyzny. Do dzisiaj nie ustalono twórcy tych niezwykle pięknych strof i oryginalnej melodii. Domniemane autorstwo przypisywane bywa Stanisławowi Magierskiemu (1904–1957) z Lublina, Bronisławowi Królowi (1916) z lwowskiej grupy poetyckiej „Żagiew” i Krystynie Krahelskiej (1914–1944).\end{info}

\begin{lyrics}[longestline={Dziś do ciebie przyjść nie mogę,}]

\firstwords{Dziś do ciebie przyjść nie mogę},\\*
Zaraz idę w nocy mrok.\\*
Nie wyglądaj za mną oknem,\\*
W mgle utonie próżno wzrok.

Po cóż ci, kochanie, wiedzieć,\\*
Że do lasu idę spać,\\*
Dłużej tu nie mogę siedzieć,\\*
Na mnie czeka leśna brać.

Księżyc zaszedł, hen, za lasem,\\*
We wsi gdzieś szczekają psy,\\*
A nie pomyśl sobie czasem,\\*
Że do innej tęskno mi.

Kiedy wrócę, dziś do ciebie,\\*
Może w dzień, a może w noc,\\*
Dobrze będzie nam jak w niebie,\\*
Pocałunków dasz mi moc.

Gdy nie wrócę, niechaj z wiosną\\*
Rolę moją sieje brat,\\*
Kości moje mchem porosną\\*
I użyźnią ziemi szmat.

W pole wyjdź pewnego ranka,\\*
Na snop żyta dłonie złóż\\*
I ucałuj, jak kochanka.\\*
Ja żyć będę w kłosach zbóż.
\end{lyrics}



\song[lyricsyear={1980},music={Natan Tenenbaum},lyrics={Przemysław Gintrowski}]
{Modlitwa o wschodzie słońca}
\begin{info}Napisana w 1980 roku i śpiewana podczas strajków, manifestacji i koncertów "Solidarności" jako swego rodzaju credo tego ruchu, deklaracji. \end{info}

\begin{lyrics}[longestline={Lecz chroń mnie, Panie, od pogardy}]

\firstwords{Każdy Twój wyrok przyjmę twardy}\\*
Przed mocą Twą się ukorzę.\\*
Lecz chroń mnie, Panie, od pogardy\\*
Od nienawiści strzeż mnie, Boże.

Wszak Tyś jest niezmierzone dobro\\*
Którego nie wyrażą słowa.\\*
Więc mnie od nienawiści obroń\\*
I od pogardy mnie zachowaj.

\begin{markverses}[atwidthof={Co postanowisz, niech się ziści.}]%
Co postanowisz, niech się ziści.\\*
Niechaj się wola Twoja stanie,\\*
Ale zbaw mnie od nienawiści\\*
Ocal mnie od pogardy, Panie.
\end{markverses}
\end{lyrics}



\song
{Płynie Wisła, płynie}
\begin{info}Pieśń ta jest anonimową odmianą słowną i melodyczną napisanego przez E. Wasilewskiego w 1840 r. krakowiaka "Od południa stoi". Odegrała ona wielką rolę wychowawczą w czasach niewoli, okupacji i zniewolenia komunistycznego.\end{info}

\begin{lyrics}[multicol=true,longestline={,,Ojcze nasz'' i ,,Zdrowaś''}]

\firstwords{Płynie Wisła, płynie}\\*
Po polskiej krainie,\\*
Po polskiej krainie.\\*
\begin{markverses}[atwidthof={Zobaczyła Krakówmm}]%
Zobaczyła Kraków,\\*
pewnie go nie minie.
\end{markverses}

Zobaczyła Kraków,\\*
Wnet go pokochała,\\*
Wnet go pokochała.\\*
\begin{markverses}[atwidthof={Zobaczyła Krakówmm}]%
A w dowód miłości\\*
wstęgą opasała.
\end{markverses}

Chociaż się schowała\\*
W Niepołomskie lasy,\\*
W Niepołomskie lasy.\\*
\begin{markverses}[atwidthof={Zobaczyła Krakówmm}]%
I do morza wpada,\\*
płynie jak przed czasy.
\end{markverses}

Nad moją kolebą\\*
Matka się schylała,\\*
Matka się schylała.\\*
\begin{markverses}[atwidthof={Zobaczyła Krakówmm}]%
I po polsku pacierz\\*
mówić nauczała.
\end{markverses}

,,Ojcze nasz'' i ,,Zdrowaś''\\*
I ,,Skład Apostolski'',\\*
I ,,Skład Apostolski''.\\*
\begin{markverses}[atwidthof={Zobaczyła Krakówmm}]%
Bym do samej śmierci\\*
kochał naród polski.
\end{markverses}

Bo ten naród polski\\*
Ma ten urok w sobie,\\*
Ma ten urok w sobie.\\*
\begin{markverses}[atwidthof={Zobaczyła Krakówmm}]%
Kto go raz pokochał,\\*
nie zapomni w grobie.
\end{markverses}

Abym gdy dorosnę\\*
Wziął Polkę za żonę,\\*
Wziął Polkę za żonę.\\*
\begin{markverses}[atwidthof={Zobaczyła Krakówmm}]%
Bo tylko Polakom\\*
Laszki przeznaczone.
\end{markverses}

Niech Francuz Francuzkę\\*
Niemiec kocha Niemkę,\\*
Niemiec kocha Niemkę.\\*
\begin{markverses}[atwidthof={Zobaczyła Krakówmm}]%
Ja zaś wolę Polkę,\\*
niźli cudzoziemkę.
\end{markverses}

I to wszystko razem\\*
Od matki słyszałem,\\*
Od matki słyszałem.\\*
\begin{markverses}[atwidthof={Zobaczyła Krakówmm}]%
Czego nie zapomnę\\*
jak nie zapomniałem.
\end{markverses}

Płynie Wisła płynie,\\*
Po polskiej krainie,\\*
Po polskiej krainie.\\*
\begin{markverses}[atwidthof={Zobaczyła Krakówmm}]%
A dopóki płynie\\*
Polska nie zaginie.
\end{markverses}
\end{lyrics}



\song[alt={Pieśń o żołnierzu tułaczu}]
{Idzie żołnierz borem, lasem}
\begin{info}Jest to jedna z najstarszych piosenek żołnierskich. Powstała najprawdopodobniej w czasach napoleońskich, lecz geneza jej powstania może także sięgać wcześniejszych lat. Najwcześniejszy jej zapis pochodzi z 1584 r., ale niektórzy badacze przypuszczają, że jej narodziny sięgają czasów bitwy pod Warną (1444 r.)\end{info}

\begin{lyrics}[longestline={Z głodu czasem, z głodu czasem}]

\firstwords{Idzie żołnierz}\\*
Borem lasem, borem lasem,\\*
Przymierając\\*
Z głodu czasem, z głodu czasem.

Suknia na nim\\*
Nie blakuje, nie blakuje,\\*
Wiatr dziurami\\*
Przelatuje, przelatuje.

Chustka czarna\\*
Jest za pasem, jest za pasem,\\*
Ale i tej\\*
Pusto czasem, pusto czasem.

Chociaż żołnierz\\*
Obszarpany, obszarpany,\\*
Przecież ujdzie\\*
Między pany, między pany

Trzeba by go\\*
Obdarować obdarować,\\*
Soli, chleba\\*
Nie żałować nie żałować.

Wtenczas żołnierza\\*
Szanują, szanują,\\*
Kiedy trwogę\\*
Na się czują, na się czują.

Zapłaćże mu\\*
Jezu z nieba, Jezu z nieba,\\*
Boć go pilna\\*
Jest potrzeba, jest potrzeba.
\end{lyrics}



\song[lyricsyear={1914-1918},music={Mieczysław Kozar-Słobódzki},lyrics={Jan Lankau, Kazimierz M. Wroczyński}]
{Białe róże}
\begin{info}Obszerny komentarz o historii piosenki odnaleźć można na http://bibliotekapiosenki.pl/Biale_roze.\end{info}

\begin{lyrics}[longestline={Nimeś próg przestąpił, kwiat na ziemi zwiądł.}]

\firstwords{Rozkwitały pąki białych róż},\\*
Wróć, Jasieńku, z tej wojenki już,\\*
\begin{markverses}%
Wróć, ucałuj, jak za dawnych lat,\\*
Dam ci za to róży najpiękniejszy kwiat.
\end{markverses}

Kładłam ci ja idącemu w bój,\\*
Białą różę na karabin twój,\\*
\begin{markverses}%
Nimeś odszedł, mój Jasieńku, stąd,\\*
Nimeś próg przestąpił, kwiat na ziemi zwiądł.
\end{markverses}

Ponad stepem nieprzejrzana mgła,\\*
Wiatr w burzanach cichuteńko łka.\\*
\begin{markverses}%
Przyszła zima, opadł róży kwiat,\\*
Poszedł w świat Jasieńko, zginął za nim ślad.
\end{markverses}

Już przekwitły pąki białych róż,\\*
Przeszło lato, jesień, zima już,\\*
\begin{markverses}%
Cóż ci teraz dam, Jasieńku, hej,\\*
Gdy z wojenki wrócisz do dziewczyny swej?
\end{markverses}

W pustym polu zimny wicher dmie,\\*
Już nie wróci twój Jasieńko, nie,\\*
\begin{markverses}%
Śmierć okrutna zbiera krwawy łup,\\*
Zakopali Jasia twego w ciemny grób.
\end{markverses}

Jasieńkowi nic nie trzeba już,\\*
Bo mu kwitną pęki białych róż,\\*
\begin{markverses}%
Tam pod jarem, gdzie w wojence padł,\\*
Rozkwitł na mogile białej róży kwiat.
\end{markverses}

Nie rozpaczaj, lube dziewczę, nie,\\*
W polskiej ziemi nie będzie mu źle.\\*
\begin{markverses}%
Policzony będzie trud i znój,\\*
Za Ojczyznę poległ ukochany twój.
\end{markverses}
\end{lyrics}



\song[music={Tadeusz Sygietyński},lyrics={Konstanty Ildefons Gałczyński}]
{Ukochany kraj}

\begin{lyrics}[longestline={Murarz, żołnierz, cieśla, zdun, inżynier}]

\firstwords{Wszystko tobie ukochana ziemio},\\*
Nasze myśli wciąż przy tobie są,\\*
Tobie lotnik tryumf nad przestrzenią,\\*
A robotnik daje dwoje rąk.\\*
\smallskip
Ty przez serca nam jak Wisła płyniesz,\\*
Brzmi jak rozkaz twój potężny głos;\\*
Murarz, żołnierz, cieśla, zdun, inżynier\\*
Wykuwamy twój szczęśliwy los.

\begin{chorus}
\chorusfirstwords{Ukochany kraj, umiłowany kraj},\\*
Ukochane i miasta i wioski\\*
Ukochany kraj, umiłowany kraj,\\*
Ukochany, jedyny nasz, polski.\\*
Ukochany kraj, umiłowany kraj,\\*
Ukochana i ziemia i nazwa\\*
Ukochany kraj, umiłowany kraj,\\*
Nasza droga i słońce i gwiazda.
\end{chorus}

My trudności wszystkie pokonamy,\\*
Żaden wróg nie złamie hartu w nas,\\*
W słońce jutra otworzymy bramy,\\*
Rozśpiewamy, rozświecimy czas.\\*
\smallskip
To dla ciebie najgorętsze słowa,\\*
Wszystkie serca, siła wszystkich rąk,\\*
To dla ciebie, piękna i ludowa,\\*
Każdy dzień i każdy nowy dom.

\chorusref
\end{lyrics}



\song[lyricsyear={1944},musicyear={1944},music={Alfred L. Schütz},lyrics={Feliks Konarski}]
{Czerwone maki}

\begin{lyrics}[longestline={Czerwieńsze będą, bo z polskiej wzrosną krwi.}]

\firstwords{Czy widzisz te gruzy na szczycie}?\\*
Tam wróg twój się kryje jak szczur!\\*
Musicie, musicie, musicie\\*
Za kark wziąć i strącić go z chmur!\\*
I poszli szaleni zażarci,\\*
I poszli zabijać i mścić,\\*
I poszli jak zawsze uparci,\\*
Jak zawsze za honor się bić.

\begin{chorus}
\chorusfirstwords{Czerwone maki na Monte Cassino}\\*
Zamiast rosy piły polską krew.\\*
Po tych makach szedł żołnierz i ginął,\\*
Lecz od śmierci silniejszy był gniew.\\*
Przejdą lata i wieki przeminą,\\*
Pozostaną ślady dawnych dni\\*
I tylko maki na Monte Cassino\\*
Czerwieńsze będą, bo z polskiej wzrosną krwi.
\end{chorus}

Runęli przez ogień, straceńcy,\\*
niejeden z nich dostał i padł,\\*
jak ci z Somosierry szaleńcy,\\*
Jak ci spod Rokitny sprzed lat.\\*
Runęli impetem szalonym,\\*
I doszli. I udał się szturm.\\*
I sztandar swój biało czerwony\\*
Zatknęli na gruzach wśród chmur,

\chorusref

Czy widzisz ten rząd białych krzyży?\\*
Tam Polak z honorem brał ślub.\\*
Idź naprzód, im dalej, im wyżej,\\*
Tym więcej ich znajdziesz u stóp.\\*
Ta ziemia do Polski należy,\\*
Choć Polska daleko jest stąd,\\*
Bo wolność krzyżami się mierzy,\\*
Historia ten jeden ma błąd.

\chorusref
\end{lyrics}



\song[musicyear={1980},music={Mieczysław Cholewa},lyrics={Krzysztof Dowgiałło}]
{Ballada o Janku Wiśniewskim}
\begin{info}Ballada opisująca zastrzelenie Zbyszka Godlewskiego 17 grudnia 1970 w Gdyni, a w szerszym kontekście wydarzenia Grudnia 1970.\end{info}

\begin{lyrics}[longestline={Świat się dowiedział, nic nie powiedział}]

\firstwords{Chłopcy z Grabówka, chłopcy z Chyloni}\\*
Dzisiaj milicja użyła broni.\\*
Dzielnieśmy stali, celnie rzucali\\*
Janek Wiśniewski padł.

Na drzwiach ponieśli go Świętojańską\\*
Naprzeciw glinom, naprzeciw tankom.\\*
Chłopcy stoczniowcy pomścijcie druha\\*
Janek Wiśniewski padł.

Huczą petardy, ścielą się gazy\\*
Na robotników sypią się razy.\\*
Padają dzieci, starcy, kobiety\\*
Janek Wiśniewski padł.

Jeden zraniony, drugi pobity\\*
Krwi się zachciało słupskim bandytom.\\*
To partia strzela do robotników\\*
Janek Wiśniewski padł.

Krwawy Kociołek, to kat Trójmiasta\\*
Przez niego giną starcy, niewiasty.\\*
Poczekaj draniu, my cię dostaniem\\*
Janek Wiśniewski padł.

Stoczniowcy Gdyni, stoczniowcy Gdańska\\*
Idźcie do domu, skończona walka.\\*
Świat się dowiedział, nic nie powiedział\\*
Janek Wiśniewski padł.

Nie płaczcie matki, to nie na darmo\\*
Nad stocznią sztandar z czarną kokardą.\\*
Za chleb i wolność, i nową Polskę\\*
Janek Wiśniewski padł.
\end{lyrics}



\song[lyricsyear={1978},music={Lluís Llach},lyrics={Jacek Kaczmarski}]
{Mury}

\begin{lyrics}[longestline={Śpiewali więc, klaskali w rytm, jak wystrzał poklask ich brzmiał,}]

\firstwords{On natchniony i młody był}, ich nie policzyłby nikt\\*
On im dodawał pieśnią sił, śpiewał że blisko już świt.\\*
Świec tysiące palili mu, znad głów podnosił się dym,\\*
Śpiewał, że czas by runął mur\ldots\\*
Oni śpiewali wraz z nim:

\begin{chorus}
\chorusfirstwords{Wyrwij murom zęby krat}!\\*
Zerwij kajdany, połam bat!\\*
A mury runą, runą, runą\\*
I pogrzebią stary świat!
\end{chorus}

Wkrótce na pamięć znali pieśń i sama melodia bez słów\\*
Niosła ze sobą starą treść, dreszcze na wskroś serc i głów.\\*
Śpiewali więc, klaskali w rytm, jak wystrzał poklask ich brzmiał,\\*
I ciążył łańcuch, zwlekał świt\ldots\\*
On wciąż śpiewał i grał:

\chorusref

Aż zobaczyli ilu ich, poczuli siłę i czas,\\*
I z pieśnią, że już blisko świt szli ulicami miast;\\*
Zwalali pomniki i rwali bruk -- Ten z nami! Ten przeciw nam!\\*
Kto sam ten nasz najgorszy wróg!\\*
A śpiewak także był sam.

Patrzył na równy tłumów marsz,\\*
Milczał wsłuchany w kroków huk,\\*
A mury rosły, rosły, rosły\\*
Łańcuch kołysał się u nóg\ldots

Patrzy na równy tłumów marsz,\\*
Milczy wsłuchany w kroków huk,\\*
A mury rosną, rosną, rosną\\*
Łańcuch kołysze się u nóg\ldots
\end{lyrics}



\song[lyricsyear={1911-12},alt={Hymn harcerski},music={autor nieznany\footnote{Na melodię \textit{Na barykady}.}},lyrics={Ignacy Kozielewski,  Olga Drahonowska-Małkowska\footnote{I. Kozielewski jest autorem wiersza, na którym oparte są zwrotki pieśni. Natomiast O.~Drahonowska-Małkowska napisała refren oraz ostateczne opracowanie tekstu.}}]
{Wszystko, co nasze}

\begin{lyrics}[longestline={I wszystko wstanie, w krąg się rozszermierzy,}]

\firstwords{Wszystko co nasze Polsce oddamy},\\*
W niej tylko życie więc idziem żyć.\\*
Świty się bielą, otwórzmy bramy,\\*
Rozkaz wydany, wstań w słońce idź!

\begin{chorus}
\chorusfirstwords{Ramię pręż, słabość krusz},\\*
Ducha tęż, Ojczyźnie miłej służ.\\*
Na jej zew, w bój czy trud\\*
Pójdzie rad harcerzy polskich ród,\\*
Harcerzy polskich ród!
\end{chorus}

Czynem bogaci, myślą skrzydlaci\\*
z płomiennych serc uczyńmy grot.\\*
Naprzód wytrwale! Śmiało! Zuchwale!\\*
W podniebne szlaki skierujmy lot.

\chorusref

Po ziemi naszej roześlem harcerzy,\\*
pobudka zabrzmi: Zbudź się! Prawdzie służ!\\*
I wszystko wstanie, w krąg się rozszermierzy,\\*
by Matkę Polskę ochronić od burz!

\chorusref
\end{lyrics}



\song[lyricsyear={luty 1863},alt={Pieśń z obozu Jeziorańskiego},music={Alfred Bojarski},lyrics={Wincenty Pol}]
{Sygnał}
\begin{info}Pieśń powstała jako akt pożegnania mężczyzn odchodzących do partyzantki. Początkowo utwór wykonywali tylko powstańcy w obozie Antoniego Jeziorańskiego. Szybko jednak zdobył powszechną popularność w oddziałach na terenie wszystkich zaborów\end{info}

\begin{lyrics}[longestline={W krwawym polu srebrne ptaszę,}]

\firstwords{W krwawym polu srebrne ptaszę},\\*
Poszli w boje chłopcy nasze.

\begin{chorus}
\chorusfirstwords{Hu, ha! Krew gra}!\\*
Duch gra! Hu, ha\\*
Niechaj Polska zna,\\*
Jakich synów ma.
\end{chorus}

Obok Orła znak Pogoni,\\*
Poszli nasi w bój bez broni.

\begin{chorus}
\chorusfirstwords{Hu, ha! Krew gra}!\\*
Duch gra! Hu,ha!\\*
Matko-Polsko żyj!\\*
Jezus, Maria, bij!
\end{chorus}

Naszym braciom dopomagaj,\\*
Nieprzyjaciół naszych smagaj.

\begin{chorus}
\chorusfirstwords{Hu, ha! Wiatr gra}!\\*
Krew gra! Hu, ha!\\*
Niechaj Polska zna,\\*
Jakich synów ma!
\end{chorus}
\end{lyrics}



\song[musicyear={1944},music={Andrzej Panufnik},lyrics={Stanisław R. Dobrowolski}]
{Warszawskie dzieci}

\begin{lyrics}[longestline={Gdy padnie rozkaz Twój, poniesiem wrogom gniew!}]

\firstwords{Nie złamie wolnych żadna klęska},\\*
Nie strwoży śmiałych żaden trud --\\*
Pójdziemy razem do zwycięstwa,\\*
Gdy ramię w ramię stanie lud.

\begin{chorus}
\chorusfirstwords{Warszawskie dzieci, pójdziemy w bój},\\*
Za każdy kamień Twój, Stolico, damy krew!\\*
Warszawskie dzieci, pójdziemy w bój,\\*
Gdy padnie rozkaz Twój, poniesiem wrogom gniew!
\end{chorus}

Powiśle, Wola i Mokotów,\\*
Ulica każda, każdy dom --\\*
Gdy padnie pierwszy strzał, bądź gotów,\\*
Jak w ręku Boga złoty grom.

\chorusref

Od piły, dłuta, młota, kielni --\\*
Stolico, synów swoich sław,\\*
Że stoją wraz przy Tobie wierni\\*
Na straży Twych żelaznych praw.

\chorusref

Poległym chwała, wolność żywym,\\*
Niech płynie w niebo dumny śpiew,\\*
Wierzymy, że nam Sprawiedliwy,\\*
Odpłaci za przelaną krew.

\chorusref
\end{lyrics}



\song[lyricsyear={1969},music={Wojciech Kilar},lyrics={Jerzy Lutowski}]
{Pieśń o małym rycerzu}

\begin{lyrics}[longestline={Wstań, unieś głowę, wsłuchaj się w słowa}]

\firstwords{W stepie szerokim, którego okiem}\\*
Nawet sokolim nie zmierzysz,\\*
Wstań, unieś głowę, wsłuchaj się w słowa\\*
Pieśni o małym rycerzu.

Choć mały ciałem, rębacz wspaniały\\*
Wyrósł nad pierwsze szermierze\\*
I wieki całe będą śpiewały\\*
Pieśni o małym rycerzu.

Ty, któryś w boju i ty, coś w znoju\\*
I ty, co liczysz i mierzysz,\\*
Wstań, unieś głowę, wsłuchaj się w słowa\\*
Pieśni o małym rycerzu.
\end{lyrics}



\song[music={autor nieznany\footnote{Melodia oparta na fragmencie marsza \textit{Pożegnanie Słowianki} skomponowanego w 1912 przez Wasyla Agapkina.}},lyrics={Roman Ślęzak}]
{Rozszumiały się wierzby płaczące}

\begin{lyrics}[longestline={Śpiew na ustach, spokojna twarz i wzrok.}]

\firstwords{Rozszumiały się wierzby płaczące},\\*
Rozpłakała się dziewczyna w głos,\\*
Od łez oczy podniosła błyszczące,\\*
Na żołnierski, na twardy życia los.

\begin{chorus}
\chorusfirstwords{Nie szumcie, wierzby, nam},\\*
Żalu, co serce rwie,\\*
Nie płacz, dziewczyno ma,\\*
Bo w partyzantce nie jest źle.\\*
Do tańca grają nam\\*
Granaty, wisów szczęk,\\*
Śmierć kosi niby łan,\\*
Lecz my nie wiemy, co to lęk.
\end{chorus}

Czy to deszcz czy słoneczna spiekota,\\*
Wszędzie słychać miarowy, równy krok,\\*
To na bój idzie leśna piechota,\\*
Śpiew na ustach, spokojna twarz i wzrok.

\chorusref

I choć droga się nasza nie kończy,\\*
Choć nie wiemy, gdzie wędrówki kres,\\*
Ale pewni jesteśmy zwycięstwa,\\*
Bo przelano już tyle krwi i łez.

\chorusref
\end{lyrics}



\song[lyricsyear={1966},alt={Ballada o pancernych},musicyear={1966},music={Adam Walaciński},lyrics={Agnieszka Osiecka}]
{Deszcze niespokojne}

\begin{lyrics}[longestline={\vin ,,Rudy'' i nasz pies.}]

\firstwords{Deszcze niespokojne}\\*
potargały sad,\\*
a my na tej wojnie\\*
ładnych parę lat.\\*
\smallskip
\vin Do domu wrócimy,\\*
\vin w piecu napalimy,\\*
\vin nakarmimy psa.\\*
\vin Przed nocą zdążymy,\\*
\vin tylko zwyciężymy,\\*
\vin a to ważna gra!

Na niebie obłoki,\\*
po wsiach pełno bzu,\\*
gdzież ten świat daleki,\\*
pełen dobrych snów.\\*
\smallskip
\vin Powrócimy wierni\\*
\vin my czterej pancerni,\\*
\vin ,,Rudy'' i nasz pies.\\*
\vin My czterej pancerni\\*
\vin powrócimy wierni\\*
\vin po wiosenny bez.
\end{lyrics}



\song[lyricsyear={1908},musicyear={1910},music={Feliks Nowowiejski},lyrics={Maria Konopnicka}]
{Rota}

\begin{lyrics}[longestline={Nie będzie Niemiec pluł nam w twarz}]

\firstwords{Nie rzucim ziemi, skąd nasz ród},\\*
Nie damy pogrześć mowy!\\*
Polski my naród, polski lud,\\*
Królewski szczep Piastowy,\\*
Nie damy, by nas zgnębił wróg\ldots\\*
\markverse[noalign=true]{Tak nam dopomóż Bóg!}

Do krwi ostatniej kropli z żył\\*
Bronić będziemy Ducha,\\*
Aż się rozpadnie w proch i pył\\*
Krzyżacka zawierucha.\\*
Twierdzą nam będzie każdy próg\ldots\\*
\markverse[noalign=true]{Tak nam dopomóż Bóg!}

Nie będzie Niemiec pluł nam w twarz\\*
Ni dzieci nam germanił.\\*
Orężny wstanie hufiec nasz,\\*
Duch będzie nam hetmanił,\\*
Pójdziem, gdy zabrzmi złoty róg\ldots\\*
\markverse[noalign=true]{Tak nam dopomóż Bóg!}

Nie damy miana Polski zgnieść\\*
Nie pójdziem żywo w trumnę.\\*
Na Polski imię, na Jej cześć\\*
Podnosim czoła dumne,\\*
Odzyska ziemię dziadów wnuk\ldots\\*
\markverse[noalign=true]{Tak nam dopomóż Bóg!}
\end{lyrics}



\song
{Teraz jest wojna}

\begin{lyrics}[multicol=true, longestline={Bo pod paltem schowana rąbanka.}]
\firstwords{Na dworze jest mrok},\\*
W pociągu jest tłok,\\*
Zaczyna się więc sielanka.\\*
On objął ją wpół,\\*
Ona gruba jak wół,\\*
Bo pod paltem schowana rąbanka.\\*
\medskip
Spod serca kap, kap,\\*
Słonina i schab,\\*
Do tego dwa balerony.\\*
Gdy on się zachwyca,\\*
Wypada polędwica\\*
I boczek nieosolony.

\begin{chorus}
\chorusfirstwords{Teraz jest wojna},\\*
Kto handluje, ten żyje.\\*
Jak sprzedam rąbankę,\\*
Słoninę, kaszankę,\\*
To bimbru się też napiję.
\end{chorus}

Gdy on się przekona,\\*
Czym handluje ona,\\*
To pociąg na stacji staje.\\*
Żandarmi wsiadają,\\*
Wszystko wygrużają,\\*
Co tylko im się daje.\\*
\medskip
Ten szuka w kieszeni,\\*
Ten szuka na ziemi,\\*
A inny po paczkach szpera.\\*
Zabrali słoninę,\\*
rąbankę, kaszankę,\\*
Niech weźmie ich jasna cholera!

\begin{chorus}
\chorusfirstwords{Teraz jest wojna},\\*
Kto handluje, ten żyje.\\*
Jak sprzedam rąbankę,\\*
Słoninę, kaszankę,\\*
To bimbru się też napiję.\\*
\medskip
Teraz jest wojna\\*
i z handlu ciężko się żyje.\\*
Zabrali rąbankę,\\*
Słoninę, kaszankę\\*
i bimbru się już nie napiję!
\end{chorus}

\end{lyrics}



\song[lyricsyear={marzec 1915},alt={Jedzie na Kasztance},music={Zygmunt Pomarański},lyrics={Wacław Biernacki}]
{Pieśń o Wodzu miłym}

\begin{lyrics}[longestline={Masz wierniejszych niż stal chłodna, niż stal chłodna}]

\firstwords{Jedzie, jedzie na Kasztance}, na Kasztance,\\*
Siwy strzelca strój, siwy strzelca strój.\\*
\markverse[noalign=true]{Hej, hej, komendancie, miły wodzu mój!}

Gdzie szabelka twa ze stali, twa ze stali?\\*
Przecież idziem w bój, przecież idziem w bój.\\*
\markverse[noalign=true]{Hej, hej, komendancie, miły wodzu mój!}

Gdzie twój mundur jeneralski, jeneralski\\*
Złotem naszywany, złotem naszywany?\\*
\markverse[noalign=true]{Hej, hej, komendancie, wodzu kochany!}

Masz wierniejszych niż stal chłodna, niż stal chłodna\\*
Młodych strzelców rój, młodych strzelców rój.\\*
\markverse[noalign=true]{Hej, hej, komendancie, miły wodzu mój!}

Nad lampasy i purpury, i purpury\\*
Wolisz strzelca strój, wolisz strzelca strój.\\*
\markverse[noalign=true]{Hej, hej, komendancie, miły wodzu mój!}

Ale pod tą szarą bluzą, szarą bluzą\\*
Serce ze złota, serce ze złota.\\*
\markverse[noalign=true]{Hej, hej, komendancie, serce ze złota!}

Ale błyszczą groźną wolą, groźną wolą\\*
Królewskie oczy, królewskie oczy.\\*
\markverse[noalign=true]{Hej, hej, komendancie, królewskie oczy!}

Pójdziem z tobą po zwycięstwa, po zwycięstwa\\*
Poprzez krew i znój, poprzez krew i znój.\\*
\markverse[noalign=true]{Hej, hej, komendancie, miły wodzu mój!}
\end{lyrics}



\song
{Bogurodzica}
\begin{info}Jest to najstarsza polska pieśń religijna, a zarazem najstarsza pieśń bojowa polskiego rycerstwa. Pochodzi prawdopodobnie z końca XIII w. Śpiewana była przez wojska polskie przed bitwami z Krzyżakami: pod Grunwaldem (1410 r.) i pod Dąbkami koło Nakła nad Notecią (1431 r.), przed bitwą z księciem litewskim Świdrygiełłą pod Wiłkomierzem (1435 r.), a także podczas koronacji królów i przy wydawaniu ważniejszych dekretów państwowych. Można ją więc określić jako ówczesny hymn narodowy. Funkcję tę utraciła jednakże w drugiej połowie XVI w.\end{info}

\begin{lyrics}[longestline={U twego Syna Gospodzina, Matko zwolena, Maryja!}]

\firstwords{Bogurodzica Dziewica, Bogiem sławiena Maryja},\\*
U twego Syna Gospodzina, Matko zwolena, Maryja!\\*
Zyszczy nam, spuści nam.\\*
Kyrie eleison.

Twego dziela Krzciciela, Bożycze,\\*
Usłysz głosy, napełń myśli człowiecze.\\*
Słysz modlitwę, jąż nosimy,\\*
A dać raczy, Jegoż prosimy:\\*
A na świecie zbożny pobyt,\\*
Po żywocie rajski przebyt.\\*
Kyrie eleison.
\end{lyrics}



\song[lyricsyear={1976},music={Włodzimierz Korcz},lyrics={Jan Pietrzak}]
{Żeby Polska była Polską}

\begin{lyrics}[longestline={Z głębi dziejów, z krain mrocznych,}]

\firstwords{Z głębi dziejów, z krain mrocznych},\\*
z puszcz odwiecznych, pól i stepów\\*
nasz rodowód, nasz początek,\\*
hen, od Piasta, Kraka, Lecha.\\*
\vin Długi łańcuch ludzkich istnień\\*
\vin połączonych myślą prostą:\\*
\begin{markverses}[atwidthof={\vin żeby Polska, żeby Polska,}]%
\vin żeby Polska, żeby Polska,\\*
\vin żeby Polska była Polską.
\end{markverses}

Wtedy, kiedy los nieznany\\*
rozsypywał nas po kątach,\\*
kiedy obce wiatry gnały\\*
obce orły na proporcach,\\*
\vin przy ogniskach wybuchała\\*
\vin niezmożona nuta swojska:\\*
\begin{markverses}[atwidthof={\vin żeby Polska, żeby Polska,}]%
\vin żeby Polska, żeby Polska,\\*
\vin żeby Polska była Polską.
\end{markverses}

Zrzucał uczeń portret cara,\\*
ksiądz Ściegienny wznosił modły,\\*
opatrywał wóz Drzymała,\\*
dumne wiersze pisał Norwid.\\*
\vin I kto szablę mógł utrzymać,\\*
\vin ten formował legion wojska,\\*
\begin{markverses}[atwidthof={\vin żeby Polska, żeby Polska,}]%
\vin żeby Polska, żeby Polska,\\*
\vin żeby Polska była Polską.
\end{markverses}

Matki, żony, w mrocznych izbach\\*
wyszywały na sztandarach\\*
hasło: ,,Honor i Ojczyzna''\\*
i ruszała w pole wiara.\\*
\vin I ruszała wiara w pole\\*
\vin od Chicago do Tobolska,\\*
\begin{markverses}[atwidthof={\vin żeby Polska, żeby Polska,}, marktext={\markstyle$\times4$}]%
\vin żeby Polska, żeby Polska,\\*
\vin żeby Polska była Polską!
\end{markverses}
\end{lyrics}


\song[lyricsyear={1831},music={Karol Kurpiński},lyrics={Karol Sienkiewicz\footnote{Autorem tekstu oryginalnego w języku francuskim jest Kazimierz Delaińgne.}}, beginonleft=true, alt={Warszawianka 1831 roku}]
{Warszawianka}
\begin{info}Casimir Francois Delavigne (1793–1843) był narodowym poetą francuskim. Pod wrażeniem powstania listopadowego napisał w 1831 r. wiersz „La Varsovienne“ („Warszawianka”). Melodię skomponował, po zapoznaniu się z przekładem Karola Sienkiewicza (1793–1869), wybitny kompozytor i dyrygent operowy – Karol Kurpiński (1785–1857), autor „Krakowiaków i Górali”. Warto wspomnieć, że Delavigne przybrał po wybuchu powstania listopadowego imię Casimir (Kazimierz), aby zamanifestować swoją solidarność z Polakami.\end{info}

\begin{lyrics}[longestline={Niech krwią zlane w bojach srogich,}]

\firstwords{Oto dziś dzień krwi i chwały},\\*
Oby dniem wskrzeszenia był.\\*
W gwiazdę Polski Orzeł Biały\\*
Patrząc lot swój w niebo wzbił;\\*
Słońcem lipca podniecany,\\*
Woła na nas z górnych stron:\\*
Powstań, Polsko, skrusz kajdany,\\*
Dziś twój tryumf albo zgon.

\begin{chorus}
\chorusfirstwords{Hej, kto Polak, na bagnety}!\\*
Żyj swobodo, Polsko żyj.\\*
Takim hasłem cnej podniety,\\*
\markverse[noalign=true]{Trąbo nasza, wrogom grzmij.}
\end{chorus}

Na koń! Woła kozak mściwy,\\*
Karać bunty polskich rot!\\*
Bez Bałkanów są ich niwy\\*
Wszystko jeden zgniecie grot!\\*
Stój, za Bałkan pierś ta stanie,\\*
Car wasz marzy płonny łup,\\*
Z wrogów naszych nie zostanie\\*
Na tej ziemi chyba trup.

\chorusref

Droga Polsko! Dzieci twoje\\*
Dziś szczęśliwszych doszły chwil\\*
Od tych sławnych, gdy ich boje\\*
Wieńczył Kremlin, Tybr i Nil,\\*
Lat dwadzieścia nasze męże\\*
Los po obcych grodach siał,\\*
Dziś, o matko, kto polęże,\\*
Na twem łonie będzie spał.

\chorusref

\breaklyrics

Wstań Kościuszko! Ugodź w serca,\\*
Co litością mamić śmią,\\*
Znałże litość ów morderca,\\*
Który Pragę zalał krwią.\\*
Niechaj krew tę krwią dziś spłaci,\\*
Niech nią zrosi grunt zły gość,\\*
Laur męczeński naszej braci\\*
Bujniej będzie po niej rość.

\chorusref

Tocz, Polaku, bój zacięty,\\*
Ulec musi dumny car,\\*
Pokaż jemu pierścień święty,\\*
Nieulękłych Polek dar.\\*
Niech to godło ślubów drogich\\*
Wrogom naszym wróży grób,\\*
Niech krwią zlane w bojach srogich,\\*
Nasz z wolnością świadczy ślub.

\chorusref

Wy przynajmniej, coście legli,\\*
W obcych krajach, za kraj swój,\\*
Bracia nasi z grobów zbiegli,\\*
Błogosławcie bratni bój.\\*
Bo zwyciężyć my gotowi\\*
Z trupów naszych tamę wznieść,\\*
By krok spóźnić olbrzymowi,\\*
Co chce światu pęta nieść.

\chorusref

Grzmijcie bębny, ryczcie działa,\\*
Dalej! Dzieci, w gęsty szyk!\\*
Wiedzie hufce wolność, chwała,\\*
Tryumf błyska w ostrzu pik.\\*
Leć, nasz Orle, w górnym pędzie,\\*
Sławie, Polsce, światu służ!\\*
Kto przeżyje, wolnym będzie;\\*
Kto umiera, wolnym już!

\chorusref
\end{lyrics}

\song[lyricsyear={1969},music={Krzysztof Klenczon},lyrics={Janusz Kondratowicz}]
{Biały krzyż}

\begin{lyrics}[longestline={dróg przebytych kurz, cień siwej mgły.}]

\firstwords{Gdy zapłonął nagle świat},\\*
bezdrożami szli przez śpiący las.\\*
Równym rytmem młodych serc\\*
niespokojne dni odmierzał czas.\\*
\smallskip
Gdzieś pozostał ognisk dym,\\*
dróg przebytych kurz, cień siwej mgły.\\*
Tylko w polu biały krzyż\\*
nie pamięta już, kto pod nim śpi.

\begin{chorus}
\chorusfirstwords{Jak myśl sprzed lat},\\*
jak wspomnień ślad\\*
wraca dziś\\*
pamięć o tych, których nie ma.
\end{chorus}

Żegnał ich wieczorny mrok,\\*
gdy ruszali w bój, gdy cichła pieśń.\\*
Szli, by walczyć o twój dom\\*
wśród zielonych pól, o nowy dzień!

\chorusref

Lecz nie wszystkim pomógł los\\*
wrócić z leśnych dróg, gdy kwitły bzy.\\*
\begin{markverses}[atwidthof={nie pamięta już, kto pod nim śpi.}]%
W szczerym polu biały krzyż\\*
nie pamięta już, kto pod nim śpi.
\end{markverses}
\end{lyrics}


\song[lyricsyear={1943},alt={Oka},music={Stefan Turski\footnote{Melodia została zaczerpnięta przez autora tekstu z piosenki \textit{Szumią Oleandry} z wodewilu Stefana Turskiego \textit{Lola z Ludwinowa}.}},lyrics={Leon Pasternak}]
{Szumi dokoła las}
\begin{info}Jedna z najbardziej znanych piosenek I Dywizji Wojska Polskiego im. Tadeusza Kościuszki.\end{info}

\begin{lyrics}[longestline={jest naszej Wisły brzeg.}]

\firstwords{Szumi dokoła las},\\*
czy to jawa, czy sen?\\*
\begin{markverses}%
Co ci przypomina,\\*
co ci przypomina\\*
widok znajomy ten?
\end{markverses}

Żółty wiślany piach,\\*
wioski słomiany dach,\\*
\begin{markverses}%
płynie, płynie Oka,\\*
jak Wisła szeroka,\\*
jak Wisła głęboka.
\end{markverses}

Szumi, hej, szumi las,\\*
gdzieżeś rzuciła nas?\\*
\begin{markverses}%
Dolo, dolo nasza,\\*
hej, dolo tułacza,\\*
gdzieżeś rzuciła nas?
\end{markverses}

Był już niejeden las,\\*
wiele przeszliśmy rzek,\\*
\begin{markverses}%
ale najpiękniejszy,\\*
ale najpiękniejszy\\*
jest naszej Wisły brzeg.
\end{markverses}

Skrwawiony Wisły brzeg…\\*
jak to męczy, boli,\\*
\begin{markverses}%
żal nam serce ścisnął,\\*
Wisło, nasza Wisło\\*
w niemieckiej niewoli.
\end{markverses}

Piękny jest Wisły brzeg,\\*
piękny jest Oki brzeg,\\*
\begin{markverses}%
jak szarża ułańska,\\*
od Wisły do Gdańska\\*
pójdziemy, dojdziemy.
\end{markverses}
\end{lyrics}



\song[lyricsyear={1917}]
{Ciężkie losy legionera}
\begin{info}Anonimowa piosenka śpiewana w 3 pp w Zegrzu w 1917 r.\end{info}

\begin{lyrics}[longestline={Maszerować jak przystało, raz, dwa, raz, dwa, trzy.}]

\firstwords{Ciężkie losy legionera, raz, dwa, trzy},\\*
Los go gnębi jak cholera, raz, dwa, trzy,\\*
Robić dużo, a jeść mało,\\*
Maszerować jak przystało, raz, dwa, raz, dwa, trzy.

Druh karabin ciąży w dłoni,\\*
Bagnet o łopatkę dzwoni,\\*
A przy boku ładownica\\*
I manierka, powiernica.

Komendanci rano wstają,\\*
Żołnierzowi spać nie dają,\\*
Ledwieś zdążył wciągnąć gacie,\\*
Już na pole smaruj, bracie.

Dawniej były lepsze czasy,\\*
Nie jadało się kiełbasy,\\*
Nim znów wrócą czasy syte,\\*
To odwalisz dawno kitę.

Chlebuś dają nam, człowiecze,\\*
Co się z starych trocin piecze,\\*
Rzadkiej zupki się napijesz\\*
I dzień cały o tym żyjesz.

Z czego robią naszą kawę?\\*
To zupełnie nieciekawe,\\*
Mięso zaś też drwi z człowieka,\\*
Bo rży, miauczy albo szczeka.

Rano ćwiczą nas po szwedzku,\\*
Cały dzionek po niemiecku,\\*
A po polsku jeść nam dają,\\*
Tak to, bracie, nas kiwają.

I porucznik nasz kochany,\\*
Widać też dziś niewyspany\\*
Albo obiad jadł w menaży,\\*
Bo coś blady jest na twarzy.
\end{lyrics}



\song[lyricsyear={sierpień 1944},musicyear={sierpień 1944},music={Jan K. Markowski},lyrics={Mirosław Jezierski}]
{Marsz Mokotowa}
\begin{info}Jedna z najbardziej znanych piosenek powstańczych. Śpiewana była również przez partyzantów z oddziałów walczących na Kielecczyźnie i w Krakowskiem. Słowa napisał kpr. pchor. ,,Karnisz'' – Mirosław Jezierski, żołnierz 2 plutonu WSOP (Wewnętrzna Służba Ochrony Powstania) na kwaterze oddziału przy ul. Goszczyńskiego na Mokotowie. Jednostka ta wchodziła w skład pułku Armii Krajowej „Baszta”. Twórcą muzyki był ppor. „Krzysztof” – Jan Markowski.\end{info}

\begin{lyrics}[longestline={\vin Ten pierwszy marsz niech dzień po dniu,}]

\firstwords{Nie grają nam surmy bojowe}\\*
I werble do szturmu nie warczą,\\*
Nam przecież te noce sierpniowe\\*
I prężne ramiona wystarczą.\\*
\smallskip
Niech płynie piosenka z barykad\\*
Wśród bloków, zaułków, ogrodów,\\*
Z chłopcami niech idzie na wypad,\\*
Pod rękę, przez cały Mokotów.

\vin Ten pierwszy marsz ma dziwną moc,\\*
\vin Tak w piersiach gra, aż braknie tchu,\\*
\vin Czy słońca żar, czy chłodna noc,\\*
\vin Prowadzi nas pod ogniem z luf.\\*
\smallskip
\vin Ten pierwszy marsz to właśnie zew,\\*
\vin Niech brzmi i trwa przy huku dział,\\*
\vin Batalion gdzieś rozpoczął szturm,\\*
\vin Spłynęła łza i pierwszy strzał!

Niech wiatr ją poniesie do miasta,\\*
Jak żagiew płonącą i krwawą,\\*
Niech w górze zawiśnie na gwiazdach,\\*
Czy słyszysz, płonąca Warszawo?\\*
\smallskip
Niech zabrzmi w uliczkach znajomych,\\*
W Alejach, gdzie bzy już nie kwitną,\\*
Gdzie w twierdze zmieniły się domy,\\*
A serca z zapału nie stygną!

\vin Ten pierwszy marsz ma dziwną moc,\\*
\vin Tak w piersiach gra, aż braknie tchu,\\*
\vin Czy słońca żar, czy chłodna noc,\\*
\vin Prowadzi nas pod ogniem z luf.\\*
\smallskip
\vin Ten pierwszy marsz niech dzień po dniu,\\*
\vin W poszumie drzew i w sercach drży,\\*
\vin Bez próżnych skarg i zbędnych słów,\\*
\vin To nasza krew i czyjeś łzy!
\end{lyrics}



\song[music={melodia ludowa},lyrics={piosenka ludowa\footnote{Wacław Denhoff-Czarnecki był autorem zwrotki siódmej i następnych, jak również o\-pra\-co\-wania całości. Autorstwo pierwszych sześciu zwrotek nie jest znane, przyjmuje się, że jest to piosenka ludowa.}}, beginonleft=true]
{O mój rozmarynie}

\begin{lyrics}[longestline={Sto dwadzieścia kulek, sto dwadzieścia kulek}]

\firstwords{O mój rozmarynie, rozwijaj się},\\*
O mój rozmarynie, rozwijaj się,\\*
\begin{markverses}%
Pójdę do dziewczyny, pójdę do jedynej\\*
Zapytam się.
\end{markverses}

A jak mi odpowie: nie kocham cię,\\*
A jak mi odpowie: nie kocham cię,\\*
\begin{markverses}%
Ułani werbują, strzelcy maszerują\\*
Zaciągnę się.
\end{markverses}

Dadzą mi buciki z ostrogami,\\*
Dadzą mi buciki z ostrogami,\\*
\begin{markverses}%
I siwy kabacik, i siwy kabacik\\*
Z wyłogami.
\end{markverses}

Dadzą mi konika cisawego,\\*
Dadzą mi konika cisawego,\\*
\begin{markverses}%
I ostrą szabelkę, i ostrą szabelkę\\*
Do boku mego.
\end{markverses}

Dadzą mi uniform popielaty,\\*
Dadzą mi uniform popielaty,\\*
\begin{markverses}%
Ażebym nie tęsknił, ażebym nie tęsknił\\*
Do swej chaty.
\end{markverses}

Dadzą mi karabin z polskiej stali,\\*
Dadzą mi karabin z polskiej stali,\\*
\begin{markverses}%
Sto dwadzieścia kulek, sto dwadzieścia kulek\\*
Co łeb rozwali.
\end{markverses}

\breaklyrics

Dadzą mi manierkę z gorzałczyną,\\*
Dadzą mi manierkę z gorzałczyną,\\*
\begin{markverses}%
Ażebym nie tęsknił, ażebym nie tęsknił\\*
Za dziewczyną.
\end{markverses}

A kiedy już wyjdę na wiarusa,\\*
A kiedy już wyjdę na wiarusa,\\*
\begin{markverses}%
Pójdę do dziewczyny, pójdę do jedynej\\*
Po całusa.
\end{markverses}

A gdy mi odpowie - nie wydam się,\\*
A gdy mi odpowie - nie wydam się,\\*
\begin{markverses}%
Hej, tam kule świszczą i bagnety błyszczą,\\*
Poświęcę się.
\end{markverses}

Pójdziemy z okopów na bagnety,\\*
Pójdziemy z okopów na bagnety,\\*
\begin{markverses}%
Bagnet mnie ukłuje, śmierć mnie pocałuje,\\*
Ale nie ty.
\end{markverses}

A gdy mnie przyniosą z raną w boku,\\*
A gdy mnie przyniosą z raną w boku,\\*
\begin{markverses}%
Wtedy pożałujesz, wtedy pożałujesz\\*
Z łezką w oku.
\end{markverses}

Za tę naszą ziemię skąpaną we krwi,\\*
Za tę naszą ziemię skąpaną we krwi,\\*
\begin{markverses}%
Za naszą niewolę, za nasze kajdany,\\*
Za wylane łzy.
\end{markverses}
\end{lyrics}



\song[lyricsyear={1933},music={Michał Zieliński},lyrics={Michał Zieliński}]
{Serce w plecaku}
\begin{info}1. w 2 wersjach youtubowych nie ma zwrotki "nad żołnierza nie masz pana"\end{info}

\begin{lyrics}[longestline={Żołnierz śmiał się, bo w plecaku}]

\firstwords{Z młodej piersi się wyrwało}\\*
W wielkim bólu i rozterce\\*
I za wojskiem poleciało\\*
Zakochane czyjeś serce.\\*
\smallskip
Żołnierz drogą maszerował,\\*
Nad serduszkiem się użalił,\\*
Więc je do plecaka schował\\*
I pomaszerował dalej.

\begin{chorus}
\chorusfirstwords{Tę piosenkę, tę jedyną}\\*
Śpiewam dla ciebie dziewczyno,\\*
Może także jest w rozterce\\*
Zakochane twoje serce?\\*
\smallskip
Może potajemnie kochasz\\*
I po nocach tęsknisz szlochasz?\\*
Tę piosenkę, tę jedyną,\\*
Śpiewam dla ciebie dziewczyno.
\end{chorus}

Nad Żołnierza nie masz pana,\\*
Nad karabin nie ma żony.\\*
O dziewczyno ukochana,\\*
Oczka twoje zasmucone.\\*
\smallskip
Tam po łące, po zielonej\\*
Żołnierz młody szedł na boje,\\*
A w plecaku miał czerwone\\*
Zakochane serce twoje.

\chorusref

Poszedł żołnierz na wojenkę\\*
Poprzez góry, lasy, pola\\*
I ze śmiercią szedł pod rękę,\\*
Taka jest żołnierska dola.\\*
\smallskip
I choć go trapiły wielce\\*
Kule, gdy szedł do ataku,\\*
Żołnierz śmiał się, bo w plecaku\\*
Miał w zapasie drugie serce.

\chorusref
\end{lyrics}



\song[lyricsyear={prawd. 1914},lyrics={autor nieznany\footnote{Autorstwo czasem przypisywane jest Feliksowi Gwiżdżowi.}}]
{Wojenka}
\begin{info}Niezwykle popularna piosenka legionowa, do tej pory anonimowa. Nie notowana przez śpiewniki żołnierskie z czasów I wojny światowej. Wydrukowana po raz pierwszy w śpiewniku Szula.
 \end{info}

\begin{lyrics}[longestline={żołnierze strzelają, żołnierze strzelają,}]

\firstwords{Wojenko, wojenko, cóżeś ty za pani},\\*
\begin{markverses}%
że za tobą idą, że za tobą idą\\*
chłopcy malowani?
\end{markverses}

Chłopcy malowani, sami wybierani,\\*
\begin{markverses}%
wojenko, wojenko, wojenko, wojenko,\\*
cóżeś ty za pani?
\end{markverses}

Na wojence ładnie, kto Boga uprosi,\\*
\begin{markverses}%
żołnierze strzelają, żołnierze strzelają,\\*
Pan Bóg kule nosi.
\end{markverses}

Lecą kule, lecą kule żwawo,\\*
\begin{markverses}%
która cię dogoni, która cię dogoni,\\*
to zapłacisz krwawo.
\end{markverses}

Wojenko, wojenko, cóżeś tak szalona,\\*
\begin{markverses}%
kogo ty pokochasz, kogo ty pokochasz,\\*
jeśli nie leguna.
\end{markverses}

Wojenko, wojenko, co za moc jest w tobie,\\*
\begin{markverses}%
kogo ty pokochasz, kogo ty pokochasz,\\*
w zimnym leży grobie.
\end{markverses}
\end{lyrics}



\song[music={Karol Kurpiński},lyrics={Jan N. Kamiński}]
{Polonez rycerski}

\begin{lyrics}[longestline={\vin Męstwo w sercu, śmiałość w słowie,}]

\firstwords{Gdy człek w taniec polski stanie},\\*
wąs podkręci, tupnie nogą,\\*
pierś mu rośnie, hej, mospanie,\\*
już raźniejszy, już mu błogo.\\*
\medskip
\vin Męstwo w sercu, śmiałość w słowie,\\*
\vin laur na skroni, kord przy boku,\\*
\vin w tej postawie szli ojcowie,\\*
\vin taki taniec wart widoku.

Hej, mospanie, gdy w dłoń klasnę,\\*
w koło damy się wywinę,\\*
wtedy, czując godność własną,\\*
mam po temu krok i minę.\\*
\medskip
\vin Bez urazy powiem śmiało:\\*
\vin w niesmak dla mnie wasze mody;\\*
\vin gdzież te uczty, gdzie te gody,\\*
\vin nie tak dawniej to bywało.
\end{lyrics}



\song
{Ułani, ułani}

\begin{lyrics}[longestline={Nie ma takiej wioski, nie ma takiej chatki,}]

\firstwords{Ułani, ułani, malowane dzieci},\\*
niejedna panienka za wami poleci.

\begin{chorus}
\chorusfirstwords{Hej, hej, ułani, malowane dzieci},\\*
niejedna panienka za wami poleci.
\end{chorus}

Niejedna panienka i niejedna wdowa,\\*
zobaczy ułana, kochać by gotowa.

\chorusref

Babcia umierała, jeszcze się pytała:\\*
czy na tamtym świecie, ułani, będziecie?

\chorusref

Nie ma takiej wioski, nie ma takiej chatki,\\*
gdzie by nie kochały ułana mężatki.

\chorusref

Nie ma takiej chatki ani przybudówki,\\*
gdzie by nie kochały ułana Żydówki.

\chorusref

Jedzie ułan, jedzie, konik pod nim pląsa,\\*
czapkę ma na bakier i podkręca wąsa.

\chorusref

Jedzie ułan, jedzie, szablą pobrzękuje,\\*
uciekaj, dziewczyno, bo cię pocałuje.

\chorusref

A niech pocałuje, nikt mu nie zabrania,\\*
wszak on swoją piersią Ojczyznę osłania.

\chorusref
\end{lyrics}



\song[lyrics={Tomasz Padura}]
{Hej, sokoły}

\begin{lyrics}[longestline={Hej, tam gdzieś znad czarnej wody,}]

\firstwords{Hej, tam gdzieś znad czarnej wody},\\*
Siada na koń kozak młody,\\*
Czule żegna się z dziewczyną,\\*
Jeszcze czulej z Ukrainą.

\begin{chorus}
\chorusfirstwords{Hej, hej, hej sokoły},\\*
Omijajcie góry, lasy, doły,\\*
Dzwoń, dzwoń, dzwoń dzwoneczku,\\*
Mój stepowy skowroneczku.\\*
\smallskip
Hej, hej, hej sokoły,\\*
Omijajcie góry, lasy, doły,\\*
Dzwoń, dzwoń, dzwoń dzwoneczku,\\*
Mój stepowy dzwoń, dzwoń, dzwoń!
\end{chorus}

Wiele dziewcząt jest na świecie,\\*
Lecz najwięcej w Ukrainie.\\*
Tam me serce pozostało,\\*
Przy kochanej mej dziewczynie.

\chorusref

Ona biedna tam została,\\*
Przepióreczka moja mała,\\*
A ja tutaj, w obcej stronie,\\*
Dniem i nocą tęsknię do niej.

\chorusref

Żal, żal za dziewczyną,\\*
Za zieloną Ukrainą,\\*
Żal, żal serce płacze,\\*
Że jej więcej nie zobaczę.

\chorusref

Wina, wina, wina dajcie,\\*
A jak umrę pochowajcie\\*
Na zielonej Ukrainie\\*
Przy kochanej mej dziewczynie.
\end{lyrics}



\song[lyricsyear={1914},music={melodia ludowa},lyrics={Feliks Gwiżdż}]
{Przybyli ułani pod okienko}

\begin{lyrics}[longestline={Gdy zwiedzim Warszawę, już nam pilno}]

\markverse{\firstwords{Przybyli ułani pod okienko},}\\*
\markverse{Pukają, wołają: -- Puść panienko!}

\markverse{-- O, Boże! A cóż to za wojacy?}\\*
\markverse{-- Otwieraj, nie bój się, to czwartacy.}

\markverse{Przyszliśmy tu poić nasze konie.}\\*
\markverse{Za nami piechoty pełne błonie.}

\markverse{-- O, Jezu! A dokąd Bóg prowadzi?}\\*
\markverse{-- Warszawę odwiedzić byśmy radzi.}

\markverse{Gdy zwiedzim Warszawę, już nam pilno}\\*
\markverse{Zobaczyć to stare nasze Wilno.}

\markverse{A stamtąd już droga nam gotowa}\\*
\markverse{Do serca polskości -- do Krakowa.}
\end{lyrics}



\song[lyricsyear={1917},alt={Marsz Pierwszej Brygady},lyrics={Tadeusz Biernacki, Andrzej T. Hałaciński}, beginonleft=true]
{Pierwsza Brygada}

\begin{lyrics}[longestline={Na stos rzuciliśmy -- nasz życia los,}]

\firstwords{Legiony to -- żołnierska nuta},\\*
Legiony to -- straceńców los,\\*
Legiony to -- żołnierska buta,\\*
Legiony to -- ofiarny stos!

\begin{chorus}
\chorusfirstwords{My, Pierwsza Brygada},\\*
Strzelecka gromada,\\*
Na stos rzuciliśmy -- nasz życia los,\\*
Na stos, na stos!
\end{chorus}

O, ile mąk, ile cierpienia,\\*
O, ile krwi, wylanych łez,\\*
Pomimo to -- nie ma zwątpienia,\\*
Dodawał sił -- wędrówki kres!

\chorusref

Mówili, żeśmy stumanieni,\\*
Nie wierząc nam, że chcieć -- to móc!\\*
Lecz trwaliśmy osamotnieni,\\*
A z nami był nasz drogi Wódz!

\chorusref

Inaczej się dziś zapatrują\\*
I trafić chcą do naszych dusz.\\*
I mówią, że już nas szanują,\\*
Lecz właśnie czas odwetu już!

\chorusref

\breaklyrics

Nie chcemy już od was uznania,\\*
Ni waszych mów, ni waszych łez!\\*
Już skończył się czas kołatania\\*
Do waszych serc -- do waszych kies!

\chorusref

Dziś nadszedł czas pokwitowania\\*
Za mękę serc i katusz dni.\\*
Nie chciejcie więc -- politowania,\\*
Zasadą jest: za krew -- chciej krwi!

\chorusref

Dzisiaj już my jednością silni\\*
Tworzymy Polskę -- przodków mit,\\*
Że wy w tej pracy nie dość pilni,\\*
Zostanie wam potomnych wstyd!

\chorusref

Umieliśmy w ogień zapału\\*
Młodzieńczych wiar rozniecić skry,\\*
Nieść życie swe dla ideału\\*
I swoją krew, i marzeń sny.

\chorusref

Potrafim dziś dla potomności\\*
Ostatki swych poświęcić dni.\\*
Wśród fałszów siać siew szlachetności\\*
Miazgą swych ciał, żarem swej krwi.

\chorusref
\end{lyrics}



\song[lyricsyear={4 sierpnia 1944},music={Józef Stiastny},lyrics={Józef Szczepański}, beginonleft=true]
{Pałacyk Michla}
\begin{info} Wojenny hymn harcerskiego Batalionu Parasol. Piosenka zdobyła sobie ogromną popularność, gdyż rozpowszechniał ją wśród powstańców Warszawy, zwłaszcza w Śródmieściu, Mieczysław Fogg.\end{info}

\begin{lyrics}[longestline={to jest nasz ,,Miecio'' w kółko golony hej!}]

\firstwords{Pałacyk Michla, Żytnia, Wola},\\*
bronią jej chłopcy od ,,Parasola'',\\*
choć na ,,tygrysy'' mają visy --\\*
to warszawiaki, fajne chłopaki -- są!

\begin{chorus}
\chorusfirstwords{Czuwaj wiaro i wytężaj słuch},\\*
pręż swój młody duch, pracując za dwóch!\\*
Czuwaj wiaro i wytężaj słuch,\\*
pręż swój młody duch, jak stal!
\end{chorus}

Każdy chłopaczek chce być ranny\ldots\\*
sanitariuszki -- morowe panny,\\*
i gdy cię kula trafi jaka,\\*
poprosisz pannę -- da ci buziaka, hej!

\chorusref

Z tyłu za linią dekowniki,\\*
intendentura, różne umrzyki,\\*
gotują zupę, czarną kawę --\\*
i tym sposobem walczą za sprawę, hej!

\chorusref

\breaklyrics

Za to dowództwo jest morowe,\\*
bo w pierwszej linii nadstawia głowę,\\*
a najmorowszy z przełożonych,\\*
to jest nasz ,,Miecio'' w kółko golony hej!

\chorusref

Wiara się bije, wiara śpiewa,\\*
szkopy się złoszczą, krew ich zalewa,\\*
różnych sposobów się imają,\\*
co chwilę ,,szafę'' nam posuwają, hej!

\chorusref

Lecz na nic ,,szafa'' i granaty,\\*
za każdym razem dostają baty\\*
i co dzień się przybliża chwila,\\*
że zwyciężymy! I do cywila, hej!

\chorusref
\end{lyrics}



\song[lyricsyear={prawd. 1918},alt={Maki},music={Stanisław Niewiadomski},lyrics={Kornel Makuszyński}]
{Ej dziewczyno, ej niebogo}

\begin{lyrics}[longestline={niech mnie zabiera, zabiera, zabiera.}]

\firstwords{Ej dziewczyno, ej niebogo},\\*
jakieś wojsko pędzi drogą,\\*
skryj się za ściany, skryj się za ściany\\*
skryj się, skryj.\\*
\vin Ja myślałam, że to maki,\\*
\vin że ogniste lecą ptaki,\\*
\markverse[noalign=true]{\vin a to ułani, ułani, ułani.}

Strzeż się tego, co na przedzie,\\*
co na karym koniu jedzie,\\*
oficyjera, oficyjera,\\*
strzeż się, strzeż!\\*
\vin Jeśli mu się wydam miła,\\*
\vin to nie będę się broniła,\\*
\markverse{\vin niech mnie zabiera, zabiera, zabiera.}

Serce weźmie i pobiegnie,\\*
Potem w krwawym polu legnie,\\*
Zostaniesz wdową, zostaniesz wdową,\\*
Strzeż się, strzeż!\\*
\vin Łez ja po nim nie uronię,\\*
\vin Jego serce mym zasłonię,\\*
\markverse{\vin Bóg go zachowa, zachowa, zachowa.}
\end{lyrics}

\song[lyricsyear={1939},lyrics={Julia Ryczer}]
{Dnia pierwszego września}
\begin{info}Chyba najpopularniejsza spośród piosenek stanowiących w latach okupacji hitlerowskiej repertuar grajków i śpiewaków, którzy produkowali się na ulicach, podwórkach, w tramwajach i w pociągach. \end{info}

\begin{lyrics}[longestline={Jeszcze Pan Bóg pomści taką straszną zbrodnię.}]

\firstwords{Dnia pierwszego września, roku pamiętnego}\\*
Wróg napadł na Polskę z kraju sąsiedniego.

Najwięcej się uwziął na naszą Warszawę,\\*
Warszawo kochana tyś jest miasto krwawe.

Kiedyś byłaś piękna, bogata, wspaniała,\\*
Teraz tylko kupa gruzów pozostała.

Domy popalone, szpitale zburzone,\\*
Gdzie się mają schronić ludzie poranione.

Lecą bomby z nieba brak jest ludziom chleba,\\*
Nie tylko od bomby umrzeć będzie trzeba.

Gdy biedna Warszawa w gruzach pozostała,\\*
To biedna Warszawa poddać się musiała.

I tak się broniła całe trzy tygodnie,\\*
Jeszcze Pan Bóg pomści taką straszną zbrodnię.
\end{lyrics}

\song[lyricsyear={1974},music={Jacek Kaczmarski},lyrics={Jacek Kaczmarski\footnote{Jest to wolne tłumaczenie piosenki Włodzimierza Wysockiego.}}, beginonleft=true]
{Obława}

\begin{lyrics}[longestline={Bo z trzema na raz walczy psami i trzech ran na raz krwawi.}]

\firstwords{Skulony w jakiejś ciemnej jamie} smaczniem sobie spał\\*
I spały małe wilczki dwa -- zupełnie ślepe jeszcze\\*
Wtem stary wilk przewodnik, co życie dobrze znał\\*
Łeb podniósł, warknął groźnie, aż mną szarpnęły dreszcze\\*
Poczułem nagle wokół siebie nienawistną woń\\*
Woń, która tłumi wszelki spokój, zrywa wszystkie sny\\*
Z daleka ktoś gdzieś krzyknął nagle krótki rozkaz -- goń!\\*
I z czterech stron wypadły na nas cztery gończe psy!

\begin{chorus}
\chorusfirstwords{Obława! Obława! Na młode wilki obława}!\\*
Te dzikie, zapalczywe, w gęstym lesie wychowane!\\*
Krąg śniegu wydeptany! W tym kręgu plama krwawa!\\*
Ciała wilcze kłami gończych psów szarpane!
\end{chorus}

Ten, który na mnie rzucił się, niewiele szczęścia miał\\*
Bo wpadł prosto mi na kły i krew trysnęła z rany\\*
Gdym teraz -- ile w łapach sił - przed siebie prosto gnał\\*
Ujrzałem małe wilczki dwa na strzępy rozszarpane!\\*
Zginęły ślepe, ufne tak, puszyste kłębki dwa\\*
Bezradne na tym świecie złym, nie wiedząc kto je zdławił\\*
I zginie także stary wilk, choć życie dobrze zna\\*
Bo z trzema na raz walczy psami i trzech ran na raz krwawi.

\chorusref

\breaklyrics

Wypadłem na otwartą przestrzeń, pianę z pyska tocząc,\\*
Lecz tutaj także ze wszech stron - zła mnie otacza woń!\\*
A myśliwemu co mnie dojrzał już się śmieją oczy\\*
O ręka pewna, niezawodna podnosi w górę broń!\\*
Rzucam się w bok, na oślep gnam, aż ziemia spod łap tryska\\*
I wtedy pada pierwszy strzał, co kark mi rozszarpuje\\*
Wciąż pędzę słyszę jak on klnie i krew mi płynie z pyska\\*
On strzela po raz drugi! Lecz teraz już pudłuje!

\chorusref

Wyrwałem się z obławy tej, schowałem w jakiś las,\\*
Lecz ile szczęścia miałem w tym to każdy chyba przyzna\\*
Leżałem w śniegu, jak nieżywy długi, długi czas\\*
Po strzale zaś na zawsze mi została krwawa blizna!\\*
Lecz nie skończyła się obława i nie śpią gończe psy\\*
I giną ciągle wilki młode na całym wielkim świecie\\*
Nie dajcie z siebie zedrzeć skór! Brońcie się i wy!\\*
O bracia wilcy! Brońcie się nim wszyscy wyginiecie!

\chorusref
\end{lyrics}







\song[lyricsyear={wrzesień 1942},music={autor nieznany\footnote{na melodię \textit{Co użyjem, to dla nas}}},lyrics={Anna Jachnina}]
{Siekiera, motyka}
\begin{info}Piosenka popularna na ulicach Warszawy w czasie okupacji niemieckiej podczas II wojny światowej.\end{info}

\begin{lyrics}[longestline={Siekiera, motyka, bimber, szklanka,}]

\firstwords{Siekiera, motyka, bimber, szklanka},\\*
w nocy nalot, w dzień łapanka,\\*
siekiera, motyka, światło, prąd,\\*
kiedyż oni pójdą stąd.

Siekiera, motyka, tramwaj, buda,\\*
Każdy zwiewa gdzie się uda,\\*
Siekiera, motyka, igła, nić,\\*
Już nie mamy gdzie się skryć.

\vin Już nie mamy gdzie się skryć,\\*
\vin Szwaby nam nie dają żyć.\\*
\vin Ich kultura nie zabrania\\*
\vin Robić takie polowania.

Siekiera, motyka, piłka, linka,\\*
tu Oświęcim, tam Treblinka,\\*
siekiera, motyka, światło, prąd,\\*
drałuj, draniu, wreszcie stąd.

Siekiera, motyka, styczeń, luty,\\*
Hitler z Ducem gubią buty,\\*
siekiera, motyka, linka, drut,\\*
już pan malarz jest kaput.

\vin Jak tu być i o czym śnić,\\*
\vin Hycle nam nie dają żyć.\\*
\vin Po ulicach gonią wciąż,\\*
\vin patrzą, kogo jeszcze wziąć.

Siekiera, motyka, piłka, alasz,\\*
przegrał wojnę głupi malarz,\\*
siekiera, motyka, piłka, nóż,\\*
przegrał wojnę już, już, już.
\end{lyrics}



\song[lyricsyear={sierpień 1914},alt={Kadrówka},music={melodia ludowa\footnote{Piosenka na melodię: \textit{Siwa gąska, siwa, po Dunaju pływa}}},lyrics={Wacław Łęcki, Tadeusz Ostrowski}]
{Pierwsza Kadrowa}
\begin{info}Piosenka ta powstała w czasie kilkudniowego marszu I Kompani Kadrowej, w dniach 6–12 sierpnia 1914\end{info}

\begin{lyrics}[longestline={Ale przecież dojdziem, byleby iść w nogę.}]

\firstwords{Raduje się serce, raduje się dusza},\\*
Gdy Pierwsza Kadrowa na wojenkę rusza.

\begin{chorus}
\chorusfirstwords{Oj da, oj da dana, kompanio kochana},\\*
Nie masz to jak Pierwsza, nie!
\end{chorus}

Chociaż do Warszawy mamy długą drogę,\\*
Ale przecież dojdziem, byleby iść w nogę.

\chorusref

Kiedy Moskal zdrajca drogę nam zastąpi,\\*
To kul z manlichera nikt mu nie poskąpi.

\chorusref

A gdyby on jeszcze śmiał udawać zucha,\\*
Każdy z nas bagnetem trafi mu do brzucha.

\chorusref

A gdy się szczęśliwie zakończy powstanie,\\*
To Pierwsza Kadrowa gwardyją zostanie.

\chorusref

A więc piersi naprzód, podniesiona głowa,\\*
Bośmy przecie Pierwsza Kompania Kadrowa

\chorusref
\end{lyrics}



\song[lyricsyear={ok. 1840},music={Wolfgang A. Mozart\footnote{fragment opery Don Juan}},lyrics={Gustaw Ehrenberg}]
{Gdy naród do boju}
\begin{info}Bojowa pieśń chłopska i hymn ruchu ludowego oraz rewolucyjnego, żywa zwłaszcza w szeregach lewicy. Pierwodruk tekstu ukazał się pt. Szlachta w roku 1831 w tomie Dźwięki z minionych lat z 1848 r.\end{info}

\begin{lyrics}[longestline={Gdy naród zawołał: ,,Umrzem lub zwyciężym!''}]

\firstwords{Gdy naród do boju wystąpił z orężem},\\*
Panowie o czynszach radzili,\\*
Gdy naród zawołał: ,,Umrzem lub zwyciężym!''\\*
Panowie w stolicy bawili.

\begin{chorus}
\chorusfirstwords{O, cześć wam, panowie magnaci},\\*
Za naszą niewolę, kajdany,\\*
O, cześć wam, książęta, hrabiowie, prałaci,\\*
Za kraj nasz krwią bratnią zbryzgany.
\end{chorus}

Armaty pod Stoczkiem zdobywała wiara\\*
Rękami czarnymi od pługa.\\*
Panowie w stolicy kurzyli cygara,\\*
Radzili o braciach zza Buga.

\chorusref

Lecz kiedy wybije godzina powstania,\\*
Magnatom lud ucztę zgotuje,\\*
Muzykę piekielną zaprosi do grania,\\*
A szlachta niech wtedy tańcuje.

\chorusref
\end{lyrics}



\song[lyricsyear={1918-1919},lyrics={Jerzy Braun}]
{Płonie ognisko }
\begin{info}Jedna z najpopularniejszych pieśni harcerskich.Twórca pieśni, Jerzy Braun, w momencie pisania utworu był maturzystą II Gimnazjum w Tarnowie. Drużynowy, o którym mówią słowa, to „Druh Bajdała” – Władysław Wodniecki, kierujący drużyną harcerską, do której należał autor.\end{info}

\begin{lyrics}[longestline={\vin O obrońcach naszych polskich granic,}]

\firstwords{Płonie ognisko i szumią knieje},\\*
Drużynowy jest wśród nas.\\*
Opowiada starodawne dzieje,\\*
Bohaterski wskrzesza czas.\\*
\medskip
\begin{markverses}
\vin O rycerstwie spod kresowych stanic,\\*
\vin O obrońcach naszych polskich granic,\\*
\vin A ponad nami wiatr szumi, wieje\\*
\vin I dębowy huczy las.
\end{markverses}

Już do powrotu głos trąbki wzywa,\\*
Alarmują ze wszech stron!\\*
Staje wiara w ordynku szczęśliwa,\\*
Serca biją w zgodny ton!\\*
\medskip
\begin{markverses}
\vin Każda twarz się uniesieniem płoni,\\*
\vin Każdy laskę krzepko dzierży w dłoni\\*
\vin A z młodzieńczej się piersi wyrywa\\*
\vin Pieśń potężna, pieśń jak dzwon.
\end{markverses}
\end{lyrics}



\song[lyricsyear={ok. 1918},music={Leon Łuskino},lyrics={Bolesław Lubicz-Zahorski, Leon Łuskino}]
{Piechota}

\begin{lyrics}[longestline={Nie noszą lampasów, lecz szary ich strój,..}]

\firstwords{Nie noszą lampasów, lecz szary ich strój},\\*
Nie noszą ni srebra, ni złota.\\*
\begin{markverses}%
Lecz w pierwszym szeregu podąża na bój\\*
Piechota, ta szara piechota.
\end{markverses}

\begin{chorus}
\chorusfirstwords{Maszerują chłopcy, maszerują},\\*
karabiny błyszczą, szary strój,\\*
a przed nimi drzewa salutują,\\*
bo za naszą Polskę idą w bój!
\end{chorus}

Idą, a w słońcu kołysze się stal.\\*
Dziewczęta zerkają zza plota,\\*
\begin{markverses}%
A oczy ich dumne utkwione są w dal,\\*
Piechota, ta szara piechota!
\end{markverses}

\chorusref

Nie grają im surmy, nie huczy im róg,\\*
A śmierć im pod stopy się miota,\\*
\begin{markverses}%
Lecz w pierwszym szeregu podąża na bój\\*
Piechota, ta szara piechota.
\end{markverses}

\chorusref
\end{lyrics}



\song[music={Jerzy Wasowski},lyrics={Bronisław Brok}]
{Po ten kwiat czerwony}

\begin{lyrics}[longestline={Po ten kwiat, po ten kwiat czerwony,}]

\firstwords{Żołnierz dziewczynie nie skłamie},\\*
Chociaż nie wszystko jej powie,\\*
Żołnierz zarzuci broń na ramię,\\*
Wróci -- to resztę dopowie.

\begin{chorus}
\chorusfirstwords{Wstęgą szos, miedzą pól złoconych},\\*
krętą ścieżką poprzez las,\\*
Po ten kwiat, po ten kwiat czerwony,\\*
Skoro przyszedł na to czas.
\end{chorus}

Dla tych co wiernie czekają\\*
Będą żołnierze śpiewali\\*
O tym, jak pięknie zakwitają\\*
Kwiaty czerwieńsze od malin.

\chorusref
\end{lyrics}

\song[lyricsyear={1920},alt={Hymn warmiński},musicyear={1920},music={Feliks Nowowiejski},lyrics={Maria Paruszewska}]
{O Warmio moja miła}

\begin{lyrics}[longestline={\vin Dziś polskie tam sztandary}]

\firstwords{O Warmio moja miła}\\*
Rodzinna ziemio ma.\\*
Tyś mnie do snu tuliła,\\*
Miłością pierś ma drga\\*
\smallskip
\vin Zdradziecko byłaś wzięta,\\*
\vin Bo chytry był nasz wróg\\*
\vin Niewoli srogie pęta,\\*
\vin Rozerwał dziś sam Bóg!

My Warmii wierne dzieci\\*
Kochamy ten nasz kraj,\\*
Po latach burz, zamieci\\*
Zabłysnął szczęścia raj.\\*
\smallskip
\vin Olsztyński zamek stary,\\*
\vin Krzyżactwa mieścił ród,\\*
\vin Dziś polskie tam sztandary\\*
\vin I odrodzenia cud.

Rozdarły Polskę wrogi,\\*
Przyszła niewola, znój,\\*
Lecz Biały Orzeł drogi\\*
Lot zwrócił ku niej swój!\\*
\smallskip
\vin Ojczyzno zmartwychwstała\\*
\vin Twych dzieci usłysz śpiew.\\*
\vin O Warmio, Polska cała\\*
\vin Za Ciebie odda krew!
\end{lyrics}



\song[lyricsyear={1925},alt={Marynarka wojenna},musicyear={1925},music={Adam Kowalski},lyrics={Adam Kowalski}]
{Morze, nasze morze}
\begin{info}Hymn Marynarki Wojennej.\end{info}

\begin{lyrics}[longestline={albo na dnie z honorem lec, z honorem lec.}]

\firstwords{Chociaż każdy z nas jest młody},\\*
lecz go starym wilkiem zwą.\\*
My, strażnicy polskiej wody,\\*
marynarze polscy to.

\begin{chorus}
\chorusfirstwords{Morze, nasze morze},\\*
wiernie ciebie będziem strzec.\\*
Mamy rozkaz cię utrzymać,\\*
albo na dnie, na dnie twoim lec,\\*
albo na dnie z honorem lec, z honorem lec.
\end{chorus}

Żadna siła, żadna burza,\\*
nie odbierze Gdańska nam.\\*
Nasza flota, choć nieduża,\\*
wiernie strzeże portu bram.

\chorusref
\end{lyrics}



\song[lyricsyear={koniec XIX~w.},music={melodia ludowa},lyrics={autor nieznany}]
{Marsz Polonia}
\begin{info}8 XII 1830 r. weszli do Kalisza Kosynierzy śpiewając Mazurka Dąbrowskiego, wkrótce narodziła się tam „Pieśń ułanów kaliskich” na tę samą melodię, przy czym wybrano na nowego dowódcę Dłuskiego. Ułani wykonywali zwrotki 5-6. Ostatnia zwrotka 9 została najprawdopodobniej ułożona przez zesłańców na Sybirze.\end{info}

\begin{lyrics}[longestline={Przejdziem Litwę, przejdziem Wołyń,}]

\firstwords{Rozproszeni po wszem świecie},\\*
gnani w obce wojny,\\*
zgromadziliśmy się przecie\\*
w jedno kółko zbrojne.

\begin{chorus}
\chorusfirstwords{Marsz, marsz, Polonia},\\*
nasz dzielny narodzie,\\*
odpoczniemy po swej pracy\\*
w ojczystej zagrodzie.
\end{chorus}

Z wiosną zabrzmi trąbka nasza,\\*
pocwałują konie.\\*
Sławą polskiego pałasza\\*
zabrzmią nasze błonie.

\chorusref

Przejdziem Litwę, przejdziem Wołyń,\\*
popasiem w Kijowie.\\*
Zimą przy węgierskim winie\\*
staniemy w Krakowie.

\chorusref

Od Krakowa bitą drogą\\*
do Warszawy wrócim.\\*
Co zastaniem, resztę wroga\\*
za łeb w Wisłę wrzucim.

\chorusref

Nad królewski gród zhańbiony\\*
wzleci orlę białe.\\*
Hukną działa, jękną dzwony,\\*
Polakom na chwałę.

\chorusref
\end{lyrics}



\song[lyricsyear={1921\footnote{Niektóre źródła wskazują koniec XIX wieku jako czas powstania utworu.}},alt={Jak długo na Wawelu},lyrics={autor nieznany}]
{Zwycięży Orzeł Biały}
\begin{info}Antoni Wójcicki komunikuje, że utwór został napisany w czasie plebiscytu na Śląsku.\end{info}

\begin{lyrics}[longestline={Jak długo w sercach naszych}]

\firstwords{Jak długo w sercach naszych}\\*
choć kropla polskiej krwi,\\*
jak długo w sercach naszych\\*
ojczysta miłość tkwi\ldots

\begin{chorus}
\chorusfirstwords{Stać będzie kraj nasz cały},\\*
stać będzie Piastów gród,\\*
zwycięży Orzeł Biały,\\*
zwycięży polski lud.
\end{chorus}

Jak długo na Wawelu\\*
brzmi Zygmuntowski dzwon,\\*
jak długo z gór karpackich\\*
rozbrzmiewa polski ton\ldots

\chorusref

Jak długo Wisła wody\\*
na Bałtyk będzie słać,\\*
jak długo polskie grody\\*
nad Wisłą będą stać\ldots

\chorusref
\end{lyrics}



\song[lyrics={ułani},music={melodia ludowa}, beginonleft=true]
{Żurawiejki}

\begin{lyrics}[longestline={\intertitle{1. Pułk Szwoleżerów J. Piłsudskiego}}]

\begin{chorus}
\chorusfirstwords{Lance do boju, szable w dłoń}\\*
Bolszewika goń, goń, goń!
\end{chorus}

\intertitle{1. Pułk Szwoleżerów J. Piłsudskiego}\\*
\smallskip
\firstwords{Ciesz się dzielny szwoleżerze},\\*
Masz protekcję w Belwederze.\\*
\medskip
Z adiutantów i lekarzy\\*
Ma Warszawa pułk gówniarzy.\\*
\medskip
Siedzą sobie tak w Warszawie\\*
Przy kieliszku i przy kawie.

\intertitle{2. Pułk Ułanów Grochowskich}\\*
\smallskip
Przy kieliszku koić troski,\\*
Zwykł ułanów pułk grochowski.\\*
\medskip
Lampas z gaci, płaszcz z gałganów\\*
To jest drugi pułk ułanów.\\*
\medskip
Pomną sotnie Budionnego\\*
Pułk ułanów Dwernickiego.

\intertitle{5. Pułk Ułanów Zasławskich}\\*
\smallskip
Kto zegarki w polu zbiera?\\*
To jest piąty pułk Hallera.\\*
\medskip
Wycierają wszystkie kąty\\*
To ułanów jest pułk piąty.\\*
\medskip
W krwawej szarży pod Zasławiem\\*
Pułk zasłużył się Warszawie.

\intertitle{10. Pułk Ułanów Litewskich}\\*
\smallskip
Z Litwy borów, pól i łanów,\\*
to dziesiąty pułk ułanów.\\*
\medskip
A rozkazów kto nie słucha,\\*
to dziesiąty pułk Obucha.\\*
\medskip
Jedzie ułan z dziesiątego,\\*
wyją psy na widok jego.

\breaklyrics

\intertitle{17. Pułk Ułanów Wielkopolskich}\\*
\smallskip
Czy to świta, czy to dnieje,\\*
Siedemnasty zawsze wieje.\\*
\medskip
Zbiorowisko wielkich panów,\\*
Siedemnasty pułk ułanów.\\*
\medskip
Trochę panów, trochę chamów –\\*
Siedemnasty pułk ułanów.

\intertitle{18. Pułk Ułanów Pomorskich}\\*
\smallskip
Pełen manier ładnych dworskich,\\*
Osiemnasty pułk pomorski.\\*
\medskip
Wiać przez morze na Pomorze\\*
Osiemnasty zawsze może.\\*
\medskip
Mają dupy jak z mosiądza\\*
To ułani są z Grudziądza.

\intertitle{19. Pułk Ułanów Wołyńskich}\\*
\smallskip
Dziewiętnasty to hołota,\\*
Bo na konie siada z płota.\\*
\medskip
Gwałci panny, gwałci wdowy\\*
Dziewiętnasty pułk morowy.

\intertitle{22. Pułk Ułanów Podkarpackich}\\*
\smallskip
Śmierdzą naftą, robią długi,\\*
To jest Pułk „Dwudziesty drugi”.\\*
\medskip
Tańczą świetnie i namiętnie,\\*
Panny ich całują chętnie.

\intertitle{23. Pułk Ułanów Grodzieńskich}\\*
\smallskip
W boju krepkij, w miru sławnyj\\*
,,Dwadcat trietij'' – prawosławnyj.\\*
\medskip
Wodku piju, samyj gławnyj\\*
,,Dwadcat trietyj'' – prawosławnyj.\\*
\medskip
Kto przykładem w boju świeci?\\*
To jest pułk dwudziesty trzeci!
\end{lyrics}

\song[alt={Jeszcze jeden mazur},music={melodia ludowa\footnote{Melodia pochodzi z ziemi wileńskiej.}}]
{Ostatni mazur}
\begin{info}Pieśń powstała prawdopodobnie przed bitwą pod Olszynką Grochowską. Mazur ten przez długie lata był mylnie kojarzony z wybuchem Powstania Listopadowego.\end{info}

\begin{lyrics}[longestline={Cyt, serduszko, nie płoń liczka,}]

\firstwords{Jeszcze jeden mazur dzisiaj},\\*
choć poranek świta,\\*
czy pozwoli panna Krysia,\\*
młody ułan pyta.\\*
\medskip
\vin I tak długo błaga, prosi,\\*
\vin boć to w polskiej ziemi,\\*
\begin{markverses}[atwidthof={\vin trąbka budzi, na koń woła}]%
\vin w pierwszą parę ją unosi,\\*
\vin a sto par za niemi.
\end{markverses}

On jej czule szepcze w uszko,\\*
ostrogami dzwoni,\\*
w pannie tłucze się serduszko\\*
i liczko się płoni.\\*
\medskip
\vin Cyt, serduszko, nie płoń liczka,\\*
\vin bo ułan niestały,\\*
\begin{markverses}[atwidthof={\vin trąbka budzi, na koń woła}]%
\vin o pół mili wre potyczka,\\*
\vin słychać pierwsze strzały.
\end{markverses}

Słychać strzały, głos pobudki,\\*
dalej na koń, hura!\\*
Lube dziewczę porzuć smutki,\\*
zatańczym mazura.\\*
\medskip
\vin Jeszcze jeden krąg dokoła,\\*
\vin jeden uścisk bratni,\\*
\begin{markverses}[atwidthof={\vin trąbka budzi, na koń woła}]%
\vin trąbka budzi, na koń woła,\\*
\vin mazur to ostatni!
\end{markverses}
\end{lyrics}


