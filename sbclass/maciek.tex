\documentclass[12pt,a4paper,twoside]{songbook}

\usepackage{t1enc}
\usepackage[polish]{babel}
\usepackage[utf8]{inputenc}

\usepackage[width=15cm]{geometry}

\usepackage{layout} % DEBUGGING, TESTING
\usepackage{lipsum} % DEBUGGING, TESTING
\usepackage{tikz}
\usetikzlibrary{matrix,decorations.pathreplacing}

\makeindex{fwidx}
\makeindex{titidx}

\title{ Wszystkie piosenki, które mam }


\begin{document}
\maketitle
\setlipsumdefault{1}

\song
{Dziś do ciebie przyjść nie mogę}



\begin{lyrics}[longestline={Dziś do ciebie przyjść nie mogę,}]
  \begin{tikzpicture}
    \firstwords{Dziś do ciebie przyjść nie mogę},\\
    Zaraz idę w nocy mrok,\\
    Nie wyglądaj za mną oknem,\\
    W mgle utonie próżno wzrok.
    
    Po cóż ci, kochanie wiedzieć,\\
    Że do lasu idę spać,\\
    Dłużej tu nie mogę siedzieć,\\
    Na mnie czeka leśna brać.
    
    Księżyc zaszedł hen, za lasem,\\
    We wsi gdzieś szczekają psy,\\
    Ą nie pomyśl sobie czasem,\\
    Że do innej tęskno mi.
    
    Kiedy wrócę znów do ciebie,\\
    Może w dzień, a może w noc,\\
    Dobrze będzie nam jak w niebie,\\
    Pocałunków dasz mi moc.
  \end{tikzpicture}
\end{lyrics}


\song
{Każdy wyrok twój}
\begin{lyrics}[longestline={Lecz chroń mnie, Panie, od pogardy}]
\firstwords{Każdy Twój wyrok przyjmę twardy}\\
Przed mocą Twą się ukorzę\\
Lecz chroń mnie, Panie, od pogardy\\
Od nienawiści strzeż mnie, Boże

Wszak Ty jesteś niezmierzone dobro\\
Którego nie wyrażą słowa

Więc mnie od nienawiści obroń\\
I od pogardy mnie zachowaj

Co postanowisz, niech się ziści\\
Niechaj się wola Twoja stanie\\
Ale zbaw mnie od nienawiści\\
Ocal mnie od pogardy, Panie
\end{lyrics}


\song[lyricsyear=1797,alt=Pieśń Legionów Polskich we Włoszech,lyrics=Józef Wybicki]
{Mazurek Dąbrowskiego}
\begin{info}Pierwotnie hymn, jako Pieśń Legionów Polskich we Włoszech, został napisany przez Józefa Wybickiego. Autor melodii opartej na motywach ludowego mazurka (właściwie mazura) jest nieznany. Pieśń powstała w dniach 16-19 lipca 1797 we włoskim miasteczku Reggio nell Emilia w Republice Cisalpińskiej (w dzisiejszych Włoszech). Pierwszy raz została wykonana publicznie 20 lipca 1797 roku. Tekst ogłoszono po raz pierwszy w Mantui w lutym 1799 w gazetce Dekada Legionowa.\end{info}

\begin{lyrics}[longestline={Przejdziem Wisłę, przejdziem Wartę,}]
\firstwords{Jeszcze Polska nie zginęła},\\
Kiedy my żyjemy.\\
Co nam obca przemoc wzięła,\\
Szablą odbierzemy.

\flagverse{Ref.}Marsz, marsz, Dąbrowski,\\
Z ziemi włoskiej do Polski.\\
Za twoim przewodem\\
Złączym się z narodem.

Przejdziem Wisłę, przejdziem Wartę,\\
Będziem Polakami.\\
Dał nam przykład Bonaparte,\\
Jak zwyciężać mamy.

\flagverse{Ref.}Marsz, marsz, Dąbrowski...

Jak Czarniecki do Poznania\\
Po szwedzkim zaborze,\\
Dla ojczyzny ratowania\\
Wrócim się przez morze.

\flagverse{Ref.}Marsz, marsz, Dąbrowski...

Już tam ojciec do swej Basi\\
Mówi zapłakany —\\
Słuchaj jeno, pono nasi\\
Biją w tarabany.

\flagverse{Ref.}Marsz, marsz, Dąbrowski...
\end{lyrics}


\song[music=Mieczysław Kozar-Słobódzki,lyrics={Jan Lankau, Kazimierz Mieczysław Wroczyński}]
{Białe róże}
\begin{lyrics}[longestline={Nimeś próg przestąpił, kwiat na ziemi zwiądł.}]
\firstwords{Rozkwitały pąki białych róż},\\
Wróć, Jasieńku, z tej wojenki już,\\
Wróć, ucałuj, jak za dawnych lat,\\
Dam ci za to róży najpiękniejszy kwiat.

Kładłam ci ja idącemu w bój,\\
Białą różę na karabin twój,\\
Nimeś odszedł, mój Jasieńku, stąd,\\
Nimeś próg przestąpił, kwiat na ziemi zwiądł.

Ponad stepem nieprzejrzana mgła,\\
Wiatr w burzanach cichuteńko łka.\\
Przyszła zima, opadł róży kwiat,\\
Poszedł w świat Jasieńko, zginął za nim ślad.

Już przekwitły pąki białych róż,\\
Przeszło lato, jesień, zima już,\\
Cóż ci teraz dam, Jasieńku, hej,\\
Gdy z wojenki wrócisz do dziewczyny swej?

W pustym polu zimny wicher dmie,\\
Już nie wróci twój Jasieńko, nie,\\
Śmierć okrutna zbiera krwawy łup,\\
Zakopali Jasia twego w ciemny grób.

Jasieńkowi nic nie trzeba już,\\
Bo mu kwitną pęki białych róż,\\
Tam pod jarem, gdzie w wojence padł,\\
Rozkwitł na mogile białej róży kwiat.

Nie rozpaczaj, lube dziewczę, nie,\\
W polskiej ziemi nie będzie mu źle.\\
Policzony będzie trud i znój,\\
Za Ojczyznę poległ ukochany twój
\end{lyrics}


\song[music=Alfred Schütz,lyrics=Feliks Konarski]
{Czerwone maki}
\begin{lyrics}[longestline={Czerwieńsze będą, bo z polskiej wzrosły krwi.}]
\firstwords{Czy widzisz te gruzy na szczycie}?\\
Tam wróg twój się ukrył jak szczur.\\
Musicie, musicie, musicie\\
Za kark wziąć i strącić go z chmur.\\
I poszli szaleni zażarci,\\
I poszli zabijać i mścić,\\
I poszli jak zawsze uparci,\\
Jak zawsze za honor się bić.

\flagverse{Ref.}Czerwone maki na Monte Cassino\\
Zamiast rosy piły polską krew.\\
Po tych makach szedł żołnierz i ginął,\\
Lecz od śmierci silniejszy był gniew.\\
Przejdą lata i wieki przeminą.\\
Pozostaną ślady starych dni\\
I tylko maki na Monte Cassino\\
Czerwieńsze będą, bo z polskiej wzrosły krwi.

Runęli przez ogień ,straceńcy,\\
niejeden z nich dostał i padł,\\
jak ci z Somosierry szaleńcy,\\
Jak ci spod Rokliny przed lat.\\
Runęli impetem szalonym,\\
I doszli . I udał się szturm.\\
I sztandar swój biało czerwony\\
Zatknęli na gruzach wśród chmur,

\flagverse{Ref.}Czerwone maki na Monte Cassino...

Czy widzisz ten rząd białych krzyży?\\
Tam Polak z honorem brał ślub.\\
Idź naprzód, im dalej ,im wyżej,\\
Tym więcej ich znajdziesz u stóp.\\
Ta ziemia do Polski należy,\\
Choć Polska daleko jest stąd,\\
Bo wolność krzyżami się mierzy,\\
Historia ten jeden ma błąd.

\flagverse{Ref.}Czerwone maki na Monte Cassino...
\end{lyrics}


\song[alt=Pieśń z obozu Jeziorańskiego]
{Sygnał}
\begin{lyrics}[longestline={W krwawym polu srebrne ptaszę,}]
\firstwords{W krwawym polu srebrne ptaszę},\\
Poszli w boje chłopcy nasze.

\flagverse{Ref.}Hu, ha! Krew gra!\\
Duch gra! Hu, ha\\
Niechaj Polska zna,\\
Jakich synów ma.

Obok Orła znak Pogoni,\\
Poszli nasi w bój bez broni...

\flagverse{Ref.}Hu, ha! Krew gra!\\
Duch gra! Hu,ha!\\
Matko-Polsko żyjl\\
Jezus, Maria, bij!

Naszym braciom dopomagaj,\\
Nieprzyjaciół naszych smagaj.

\flagverse{Ref.}Hu, ha! Wiatr gra!\\
Krew gra! Wiatr gra!\\
Niechaj Polska zna,\\
Jakich synów ma!
\end{lyrics}


\song
{Hej, sokoły}
\begin{lyrics}[longestline={Hej, tam gdzieś znad czarnej wody,}]
\firstwords{Hej, tam gdzieś znad czarnej wody},\\
Wsiada na koń kozak młody,\\
Czule żegna się z dziewczyną,\\
Jeszcze czulej z Ukrainą.

\flagverse{Ref.}Hej, hej, hej sokoły,\\
Omijajcie góry, lasy, doły,\\
Dzwoń, dzwoń, dzwoń dzwoneczku,\\
Mój stepowy skowroneczku.

Żal, żal za dziewczyną,\\
Za zieloną Ukrainą,\\
Żal, żal serce płacze,\\
Żal, że już jej nie zobaczę.

\flagverse{Ref.}Hej, hej, hej sokoły...

Ona biedna tam została,\\
Przepióreczka moja mała,\\
A ja tutaj, w obcej strome,\\
Dniem i nocą tęsknię do niej.

\flagverse{Ref.}Hej, hej, hej sokoły...
\end{lyrics}




\printindex{titidx}{Piosenki wg tytułów}
\printindex{fwidx}{Piosenki wg pierwszych słów}
\end{document}