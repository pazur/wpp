\song
[alt={Niech żyje Lorem!\footnote{Tu też można dawać przypisy, nawet takie długie jak ten. Czy da się jeszcze dłuższe? No pewnie!}},
 lyrics={nieznany\footnote{Miał w tym udział Cyceron}},
 lyricsyear={XVw.}
]
{Lorem ipsum}

\begin{info}
  \emph{Aby nie wyświetlać tych informacji, wystarczy dodać \texttt{noinfo} do opcji dokumentu}
  Jest to jeden z najznakomitszych tekstów Renenansu!
\end{info}

\begin{lyrics}[longestline={Lorem Lorem Lorem Lorem Lorem}]
  \firstwords{Lorem ipsum dolor sit amet},\\*
  consectetur adipisicing elit.\\*
  Proin nibh augue, suscipit a,\\*
  Cras vel lorem.

  \begin{chorus}%
    \chorusfirstwords{Niech nam żyje ,,Lorem Ipsum''},\\*
    Cieszmy się, że mam go!
  \end{chorus}
  
  Etiam pellentesque aliquet tellus.\\*
  \begin{markverses}%
    Quisque semper justo at risus.\\*
    Donec venenatis, turpis vel\\*
    hendrerit interdum,\\*
    dui ligula ultricies purus,\end{markverses}\\*
  sed posuere libero dui id orci

  \chorusref

  Etiam pellentesque aliquet tellus.\\*
  Phasellus pharetra nulla ac diam.\\*
  \begin{markverses}[marktext={można też napisać co innego}]
    Donec venenatis, turpis vel\\*
    hendrerit interdum,\\*
    dui ligula ultricies purus,\\*
    sed posuere libero dui id orci
  \end{markverses}

  \begin{chorus}[mark=false]%
    \chorusfirstwords{No i jeszcze ,,Lorem Ipsum''},\\*
    Cieszmy się, że mam go! \\*
    Tym razem bez powtórki jednakowoż!
  \end{chorus}

  Etiam pellentesque aliquet tellus.\\*
  \markverse{Phasellus pharetra nulla ac diam.}\\*
  \markverse[marktext={jeden wers też można zaznaczyć}]{Quisque semper justo at risus.}\\*
  hendrerit interdum,\\*
  dui ligula ultricies purus,\\*
  sed posuere libero dui id orci

  \chorusref
\end{lyrics}


\song
[alt={Niech żyje Lorem!\footnote{Tu też można dawać przypisy, nawet takie długie jak ten. Czy da się jeszcze dłuższe? No pewnie!}},
 lyrics={nieznany\footnote{Miał w tym udział Cyceron}},
 lyricsyear={XVw.}
]
{Lorem ipsum}

\begin{info}
  \emph{Aby nie wyświetlać tych informacji, wystarczy dodać \texttt{noinfo} do opcji dokumentu}
  Jest to jeden z najznakomitszych tekstów Renenansu!
\end{info}

\begin{lyrics}[multicol=true,longestline={Lorem Lorem Lorem Lorem Lorem}]
  \firstwords{Lorem ipsum dolor sit amet},\\*
  consectetur adipisicing elit.\\*
  Proin nibh augue, suscipit a,\\*
  Cras vel lorem.

  \begin{chorus}%
    \chorusfirstwords{Niech nam żyje ,,Lorem Ipsum''},\\*
    Cieszmy się, że mam go!
  \end{chorus}
  
  Etiam pellentesque aliquet tellus.\\*
  \begin{markverses}%
    Quisque semper justo at risus.\\*
    Donec venenatis, turpis vel\\*
    hendrerit interdum,\\*
    dui ligula ultricies purus,\end{markverses}\\*
  sed posuere libero dui id orci

  \chorusref

  Etiam pellentesque aliquet tellus.\\*
  Phasellus pharetra nulla ac diam.\\*
  \begin{markverses}[marktext={można też napisać co innego}]%
    Donec venenatis, turpis vel\\*
    hendrerit interdum,\\*
    dui ligula ultricies purus,\\*
    sed posuere libero dui id orci
  \end{markverses}

  \begin{chorus}[mark=false]%
    \chorusfirstwords{No i jeszcze ,,Lorem Ipsum''},\\*
    Cieszmy się, że mam go! \\*
    Tym razem bez powtórki jednakowoż!
  \end{chorus}

  Etiam pellentesque aliquet tellus.\\*
  \markverse{Phasellus pharetra nulla ac diam.}\\*
  \markverse[marktext={jeden wers też można zaznaczyć}]{Quisque semper justo at risus.}\\*
  hendrerit interdum,\\*
  dui ligula ultricies purus,\\*
  sed posuere libero dui id orci

  \chorusref
\end{lyrics}
